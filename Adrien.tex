%Adrien's part
\chapter{Propriété physique}

\paragraph{}
Le kevlar est un aramide. Le mot aramide vient de la contraction de l'anglais aromatic polyamide. Le kevlar possède des propriétés physiques très impressionnantes. Ce sont ces propriétés qui font que c’est un matériau utilisé dans diverses domaines. Il est apprécié pour sa résistance mécanique à la traction. Nous allons reprendre ci-dessous les principales caractéristiques du Kevlar comme sa résistance à la traction, son module d’élasticité,…



\begin{center}
\begin{tabular}{|c|c|c|c|c|}
\hline

matériau & résistance à ma traction $[MPa]$ & masse volumique $[\frac{g}{cm^3}]$& résistance spécifique $[\frac{MPa \cdot cm^3}{g}]$\\

\hline
Kelvar 49 & 3600-4100 & 1.44 & 2700\\
\hline
Carbone & 5700 & 1.5 & 3800 \\
\hline
Acier & 760 & 7.8 & 97 \\
\hline
Béton & 3 & 2.5 & 1.3 \\
\hline
\end{tabular}
\end{center}


\paragraph{}
Nous pouvons observer dans ce tableau qu’avec une résistance à la traction de plus de 3600 MPa le kevlar possède une résistance cinq fois supérieure à celle de l’acier en étant plus léger. En effet, la masse volumique de kevlar est d’1.44 g/cm3. Ce qui fait la résistance spécifique du kevlar est 28.7 fois supérieur à celui de l’acier. La résistance spécifique d’un matériau est sa résistance par rapport à sa masse volumique. Par contre, le kevlar possède une résistance spécifique inférieure à celle du carbone. Nous savons aussi qu'avec une ductilité de 3\%, le kevlar n'est pas un matériau ductile.

\paragraph{}
 Le kevlar est un matériau anisotrope. Cela veut dire que sa résistance dans le sens de la fibre n’est pas la même que perpendiculairement à la fibre. C’est à cause de cette anisotropie que le kevlar ou d’autres fibres ne sont généralement pas utilisé comme matériau seul mais comme renfort.
 
\paragraph{}
Le coefficient de dilatation thermique du kevlar est nul. Cela veut dire que quelle que soit la température, le volume du kevlar restera constant.  En plus de cela il est insensible aux produits chimiques mécaniques comme les graisses, huiles, solvants et au pétrole.

\begin{center}
\begin{tabular}{|c|c|c|c|}
\hline

matériau & ténacité $[MPa]$ & module $[GPa]$ & élongation avant rupture \\

\hline
Kelvar 29 & 2920 & 70 & 3.6 \\
\hline
Kelvar 49 & 3000 & 112 & 2 \\
\hline
Carbone & 3100 & 220 & \\
\hline
Acier & 1965 & 200 &  \\
\hline
\end{tabular}
\end{center}

\paragraph{}
Comme pouvons le constater le kevlar possède un module d’élasticité de minimum 70GPa. Même si cette fois il est inférieur au module de l’acier c’est quand même une bonne valeur.  Le kevlar à également une haute ténacité. La ténacité caractérise le comportement d’un matériau à la rupture en présence d’une entaille.

\paragraph{}
Par contre le kevlar a une très mauvaise résistance à la compression. La résistance à la compression du kevlar est égale à un dixième de sa résistance à la traction. La faible tenue mécanique en compression est généralement attribuée à une mauvaise adhérence des fibres à la matrice dans le matériau composite

\paragraph{}
L’un de ses principaux défauts est sa résistance aux UV. Nous pouvons en effet constater via le graphe ci-dessous qu’après plusieurs centaines d’heures d’expositions aux UV, le kevlar perd une bonne partie de sa résistance.


%\begin{figure}[h]
%\begin{center}
%\includegraphics[scale=0.3]{Schema/UVkevlar.png} 
%\end{center}
%\end{figure} 

\paragraph{}
Un autre défaut du kevlar est son comportement face à la chaleur. Il se décompose à partir de 450 \degre C. Une longue exposition à la température réduit la résistance à la traction, diminue son module d’élasticité et son élongation jusqu’à la rupture. La température maximum recommandée pour une utilisation du kevlar dans l’air est de 177 \degre C.



%\begin{center}
%\includegraphics[scale=1]{Schema/temperature.png} 
%\end{center}
%
%
%\begin{center}
%\includegraphics[scale=0.3]{Schema/temperature3.png} 
%\end{center}

\paragraph{}
Un autre défaut du kevlar est sa grande reprise d'humidité. Le kevlar absorbe rapidement l'humidité présente dans l'air. Le kevlar est sensibles à l'hydrolyse suite à la présence de fonctions amides. Le schéma ci-dessous montre la réaction d'une fonction amide en présence d'eau.

%\begin{figure}[h]
%\begin{center}
%\includegraphics[scale=0.65]{Schema/hydrolyse.png} 
%\caption{Mécanisme de l’hydrolyse du PPTA}
%\end{center}
%\end{figure} 

\paragraph{}
Ce mécanisme casse des chaînes au niveau des liaisons C-N de la fonction amide. Ces ruptures impliquent la formation de chaînes acide et amine. Il a été prouvé que ces ruptures de chaînes induisent une diminution de la résistance mécanique du kevlar.


%\begin{figure}[h]
%\begin{center}
%\includegraphics[scale=0.6]{Schema/hydrolyse2.png} 
%\caption{Taux de dégradation hydrolytique de la 
%résistance mécanique de fibres Kevlar 49 
%en fonction de la température à 100\% d'humidité relative. }
%\end{center}
%\end{figure} 

\paragraph{}

Le Kevlar est virtuellement non affecté par un PH de 7, c’est-à-dire un PH neutre.  Cependant, sa résistance diminue dans les milieux basiques et acides.  La résistance décroit au fur et à mesure qu’on s’éloigne du PH neutre.  Mais le Kevlar résiste mieux aux solutions basiques qu’aux solutions acides.   Voilà un schéma qui montre l’évolution de la résistance du Kevlar plongé dans une solution pendant 16h.

%Schema

Riewald et al. a réalisé une étude concernant la résistance des fibres Kevlar 29 et 49 et il en découle que les fibres perdent 1.5\% de leur résistance mécanique après un an d’immersion dans l’eau de mer.  

