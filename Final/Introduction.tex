%Intro

\section{Introduction}

	\paragraph*{} Le Kevlar est une fibre synthétique super résistante faisant partie de la famille des aramides. Ce polymère est surtout convoité pour ses propriétés exceptionnelles : résistance au feu, à la traction, au coupures,... C'est pourquoi on le retrouve dans la composition de nombreux gilets pare-balles, équipements résistants au feu, casques militaires et bien d'autres. 
	
	\paragraph*{} Le Kevlar est intégralement une invention humaine. Dans le courant des années 60, la société DuPont de Nemours mena des recherches sur des fibres plus légères mais résistantes. En 1965, Stephanie Kwolek et Herbert Blades, deux chercheurs de la société découvrirent par hasard un polymère qui, une fois filé, est quelques fois plus rigide que leurs autres produits comme le nylon. Il s'agit de la fibre dont le Kevlar est fait. Après plusieurs années, de recherches supplémentaires, d'amélioration du procédé et de réduction des coûts, le Kevlar est commercialisé en 1971. Grâce à cette découverte, Stephanie Kwolek est entrée en 1995 au \textit{United States Inventor’s Hall of Fame}. Au départ, DuPont était le seul à produire le Kevlar, mais depuis que le brevet est expiré, on retrouve d'autres producteurs sur le marché, notamment Teijin au Japon. Il n'en reste pas moins que DuPont détient toujours la plus grosse part dans le marché des fibres d'aramide. 
	
	\paragraph*{} La première partie sera consacrée à la production du Kevlar, les différentes étapes nécessaires, les procédés mis en oeuvre ainsi qu'un Flow-sheet. Ensuite viendront ses différentes propriétés physiques et chimiques, ce qui le rend si attrayant, si efficace, tout en ayant un aperçu de ses performances comparées à d'autres matériaux structuraux courants, de référence (béton, acier,...) ainsi qu'à des produits similaires (d'autres fibres). Enfin viendra l'application du Kevlar au domaine des voiles de voiliers de compétition, afin de comprendre ce qui rend ce matériaux préférable à d'autres dans ce domaine spécifique, et quelles sont ses propriétés mises en avant, ou, au contaire celles qui sont plutôt défavorable.