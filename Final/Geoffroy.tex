%Geo's part
\section{Production}


La production du Kevlar comprend deux grandes étapes :

\begin{itemize}
	\item la polymérisation, à savoir la production du polymère à partir duquel le Kevlar est fait. Le polymère est synthétisé à partir de deux monomères : la \textit{p-phénylènediamine} ainsi que le \textit{chlorure de téréphtaloyle}. Cette étape est chimique.
	\item le filage du polymère, le transformer en une fibre solide, résistante. Cette phase est mécanique, il s'agit de manipuler le polymère afin de changer ses propriétés mécaniques.
\end{itemize}
	
	
\subsection{Polymérisation}

La réaction de la première étape est une polycondensation\footnote{Durant une réaction de condensation, deux molécules, ou groupements, se combinent pour former une molécule plus grosse tout en éliminant une plus petite molécule (le sous-produit, le plus souvent de l'eau). Chaque étape d'une polycondensation est une réaction de condensation.} réalisée à basse température en milieu solvant, qui aboutit à une longue chaîne de molécules, un polymère. La température de réaction se situe entre -15 et 30\celsius ~et la solution est continuellement remuée durant la polymérisation, pouvant aller de 2 à 24 heures. Le polyamide résultant est composé de noyaux aromatiques (benzène), alternés de groupements aromatiques. Initialement, le solvant était du hexamethylphosphoramide (HMPA) mais pour des raisons de toxicité (mineures), il a été remplacé par une solution de N-méthyl-2-pyrrolidone (NMP) et de chlorure de calcium.
\begin{figure}
\centering
\includegraphics[scale=0.215]{Schema/Kevlar_chemical_synthesis.png}
\caption{Synthétisation du kevlar}
\label{kevlarsynthesis}
\end{figure}
Le sous-produit de cette réaction est de l'acide chlorhydrique, \ce{HCl}. La présence de cet acide en solution pourrait provoquer des réactions de corrosion et endommager du matériel. Il est donc nécessaire d'introduire des bases (généralement des sels de lithium et carbone) pour neutraliser l'acidité.

Après cette étape, la solution à pris la forme d'une masse visqueuse semblable à du gel. Dans cette masse, il pourrait y avoir des sels insolubles en plus de la fibre voulue. Il faut donc les éliminer avant de continuer le processus.
		
\subsection{Filage}
	
A la sortie de la première étape, les fibres sont alignées aléatoirement. Cependant, la Kevlar à un comportement nématique, comme un cristal liquide. C'est-à-dire que les fibres ont tendance à s'aligner dans la même direction, par petits paquets, il y a un certain degré d'ordre naturel. En augmentant la concentration, le degré d'alignement augmente mais n'est pas encore parfait.

\begin{wrapfigure}{R}{0.18\textwidth}
 \vspace{-20pt}
\centering
\includegraphics[scale=0.6]{Schema/kevlar_1.jpg} 
\vspace{-10pt}
\caption{Filage}
\vspace{-20pt}
\label{filage} 
\end{wrapfigure} 


Pour remédier à celà vient la deuxième étape. Les fibres sont filées, lors d'un processus de filage humide. Les fibres sont dissoutes dans de l'acide sulfurique, \ce{H_2SO_4}, et la solution passe sous pression (pour augmenter la concentration) par de très petits orifices afin d'aligner le plus possible les fibres. Le dispositif est une filière. Lors du passage de la solution à travers les minuscules trous de la filière, les fibres s'alignent dans une même direction et le résultat est ensuite refroidi dans de l'eau froide. 

A la fin de cette étape, le résultat est un long fil de Kevlar. Les fils peuvent être utilisés comme tels, comme fil de pêche ou fil à coudre. Ils peuvent aussi être tressé pour former des câbles, des cordes. Une autre manière de les utiliser est de les tisser afin de créer des bandes pouvant entrer dans la composition de gilets pare-balles, gants résistants au feu,...


Sur la figure~\ref{Flowsheet}, on peut voir la flowsheet du procédé.
%	%Définir les formes
%	\tikzstyle{block} = [rectangle, draw, fill=blue!20,text width=14em, text badly centered, rounded corners, minimum height=4em, minimum width=23em, node distance=3cm] %Etape
%	\tikzstyle{line} = [draw, ->,>=latex] %Flèche
%	\tikzstyle{inn} = [ draw, ellipse,fill=green!60, minimum height=2em, align = center] %Entrée
%	\tikzstyle{outt} = [midway, draw, ellipse,fill=red!60, minimum height=2em, align = center] %Sortie
%	\tikzstyle{inout} = [midway, draw, ellipse,shade, top color = red!70, bottom color = green!70, minimum height=2em, align = center] %Entrée/Sortie
%	\tikzstyle{values} = [rectangle, draw, fill = gray!20] %Valeur
%	\tikzstyle{final} = [midway, diamond, draw, fill=yellow!60, text width=4.5em, text badly centered, node distance=3cm, inner sep=0pt] %Produit final


\begin{center}
\begin{wrapfigure}{L}{0.25\textwidth}
 \vspace{-20pt}
\centering

	\scalebox{0.6}{
	\begin{tikzpicture}
	    %Place nodes
	    
	    %Première étape : polymérisation 
	    \node[block] (Pol)at(0,13) {\textbf{Polymérisation} \\
	    \textit{Processus chimique}
	    $$\ce{[C_6H_4(NH_2)_2 + C_8H_4Cl_2O_2]_n} $$ 
	    $$\ce{->[\ce{-2 n HCl}]}$$
	    $$\ce{[-CO-C_6H_4-CO-NH-C_6H_4-NH-]_n} $$
	    };
	    
	    %Deuxième étape : filage
	    \node[block] (Fil)at(0,6) {\textbf{Filage} \\
	    \textit{Processus mécanique}
	    \ce{[-CO-C_6H_4-CO-NH-C_6H_4-NH-]_n} \\
	    $\downarrow$ \\
	    Kevlar
	    };
	
	    \node (in1)at(-5,19) {}; %Premier réactif -4
	    \node (in2)at(5,19) {};  %Deuxième réactif 4
	    \node (out1)at(10.5,13) {}; %Rejet de HCl 9
	    	\node (pfinal)at(0,0) {};  %Produit final
	    	\node[values] (outtt)at(7.2,14) {$306,723\kilogram = 8,404\kilo\mole$}; %Valeurs sur HCl 6.46 13.75
	    		    	
	    % Draw edges
	    
	    %Premier réactif
	    \path [line] (in1) -- node[inn] {p-phénylènediamine \\ \ce{C_6H_4(NH_2)_2}} node[values, near start] {$453,782\kilogram = 4,202\kilo\mole$} (Pol); 
	    %Deuxième réactif
	    \path [line] (in2) -- node[inn] {chlorure de téréphtaloyle \\ \ce{C_8H_4Cl_2O_2}} node[values, near start] {$852,941\kilogram = 4,202\kilo\mole$}(Pol); 
	    %Entre les deux étapes
	    \path [line] (Pol) -- node[inout] {PPD-T \\ \ce{[C_6H_4(NH_2)_2 + C_8H_4Cl_2O_2]_n}} node[values, near start] {$1000\kilogram = 4,202\kilo\mole$} (Fil); 
	    %Rejet de HCl
	    \path [line] (Pol) -- node[outt] {acide chlorhydrique \\ \ce{HCl}} (out1); 
	    %Produit final
	    \path [line] (Fil) -- node[final] {\textbf{Kevlar}} node[values, pos = 0.8] {$1000 \kilogram = 4,202\kilo\mole$} (pfinal);  

	\end{tikzpicture} }
	%\vspace{-5pt}
\caption{Flowsheet}
\vspace{-30pt}
\label{Flowsheet} 
\end{wrapfigure} 
\end{center}

\subsection{Coûts}

Le Kevlar est relativement cher à produire, bien qu'il soit difficile de trouver une estimation précise des coûts de production. Ce coût élevé est dû à différents paramètres. Tout d'abord, il y a derrière le Kevlar des années de recherches et d'innovation. De plus, la réaction nécessite des équipements bien spécifiques, et, par conséquent, peu répandus et chers. Enfin, une troisième cause du haut prix du Kevlar provient de l'étape de filage. En effet, le processus requiert l'utilisation d'acide sulfurique, qui, en plus d'être très coûteux, est extrêmement dangereux. Cela hausse le coût en ajoutant encore une contrainte au niveau des équipements nécessaires, qui doivent résister à cette substance et en exigeant des mesures de sécurité supplémentaires.

Le prix de vente du Kevlar est approximativement compris entre 20 et 40\euro ~du \unit{\cubic\meter}, voire plus, tout dépend de la qualité et du type de Kevlar.