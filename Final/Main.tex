\documentclass[11pt, landscape, twocolumn]{article}
\usepackage{multicol}
\usepackage[utf8]{inputenc}
\usepackage[T1]{fontenc}      
\usepackage[francais]{babel}
\usepackage{graphicx}
\usepackage{circuitikz}
\usepackage[squaren, Gray]{SIunits}
\usepackage{sistyle}
\usepackage[autolanguage]{numprint}
\usepackage{pgfplots}
\usepackage{wrapfig}
\usepackage[a3paper]{geometry}
\usepackage{fullpage}
\usepackage{hyperref}
\usepackage{caption}
\usepackage{amsmath,amssymb,array}
\usepackage{url}
\usepackage{fancyhdr}
\usepackage{layout}
\newcommand\HUGE{\@setfontsize\Huge{38}{47}} 
\usepackage[version=3]{mhchem}
\usepackage{array} 
\usepackage{tikz}
\usetikzlibrary{arrows,shapes,positioning}
\title{Le kevlar}
\author{}
\setlength{\columnsep}{1.5cm}
\begin{document}
\twocolumn[%
%  Title and authors
   \begin{center}
     {\fontsize{100}{100}\selectfont Le kevlar}\\
      \vspace{2ex}
      Par \textsc{Thibaut \bsc{Cabo}, Adrien \bsc{Hoedenaeken}, Geoffroy \bsc{Jacquet}, Corentin \bsc{Joachim} et Léa \bsc{Paulus}}
   \end{center}
%  Abstract
%   \begin{quotation}
%      \small
%      Dans le cadre du cours LAUCE??, nous avons étudié les caractéristiques chimiques et physiques du Kevlar45 en particulier pour l'application d'une voile de bateau.
%   \end{quotation}
%   \vspace{2ex}%
]
\sffamily

%\begin{multicols}{3}
%Intro

\section{Introduction}

	\paragraph*{} Le Kevlar est une fibre synthétique super résistante faisant partie de la famille des aramides. Ce polymère est surtout convoité pour ses propriétés exceptionnelles : résistance au feu, à la traction, au coupures,... C'est pourquoi on le retrouve dans la composition de nombreux gilets pare-balles, équipements résistants au feu, casques militaires et bien d'autres. 
	
	\paragraph*{} Le Kevlar est intégralement une invention humaine. Dans le courant des années 60, la société DuPont de Nemours mena des recherches sur des fibres plus légères mais résistantes. En 1965, Stephanie Kwolek et Herbert Blades, deux chercheurs de la société découvrirent par hasard un polymère qui, une fois filé, est quelques fois plus rigide que leurs autres produits comme le nylon. Il s'agit de la fibre dont le Kevlar est fait. Après plusieurs années, de recherches supplémentaires, d'amélioration du procédé et de réduction des coûts, le Kevlar est commercialisé en 1971. Grâce à cette découverte, Stephanie Kwolek est entrée en 1995 au \textit{United States Inventor’s Hall of Fame}. Au départ, DuPont était le seul à produire le Kevlar, mais depuis que le brevet est expiré, on retrouve d'autres producteurs sur le marché, notamment Teijin au Japon. Il n'en reste pas moins que DuPont détient toujours la plus grosse part dans le marché des fibres d'aramide. 
	
	\paragraph*{} La première partie sera consacrée à la production du Kevlar, les différentes étapes nécessaires, les procédés mis en oeuvre ainsi qu'un Flow-sheet. Ensuite viendront ses différentes propriétés physiques et chimiques, ce qui le rend si attrayant, si efficace, tout en ayant un aperçu de ses performances comparées à d'autres matériaux structuraux courants, de référence (béton, acier,...) ainsi qu'à des produits similaires (d'autres fibres). Enfin viendra l'application du Kevlar au domaine des voiles de voiliers de compétition, afin de comprendre ce qui rend ce matériaux préférable à d'autres dans ce domaine spécifique, et quelles sont ses propriétés mises en avant, ou, au contaire celles qui sont plutôt défavorable.
\section{Propriétés physiques et chimiques du matériaux}
\subsection{Disposition générale de la molécule}%c'est pas un peu pourri comme titre ? :p

%	Ces monomères se lient les uns aux autres pour former des longues chaînes. Entre le groupe amide et le groupe phényle, les molécules se disposent d'une manière spéciale, les \ce{O} et les \ce{H} se disposent de part et d'autres de la chaîne.  Cela est nécessaire pour pouvoir former des liaisons hydrogènes entre les différents monomères, si les \ce{C} et le \ce{H} étaient du même côtés de la chaîne, il y aurait trop peu d'espace pour que les liaisons hydrogènes puissent se former et dès lors la molécule perdrait beaucoup de résistance.
%	
%	 On peut également ajouter que les polymères de Kevlar ont une orientation radiale\footnote{c'est à dire que la longueur de la fibre est parallèle au tissus} et que c'est un matériau anisotrope\footnote{La résistance dans le sens de la fibre n'est pas la même que perpendiculairement à la fibre}.  A cause de cette géométrie structurelle, on comprend que la reprise en traction soit bien plus grande que la reprise en compression.  
	
\begin{wrapfigure}{L}{0.12\textwidth}
 \vspace{-10pt}
\centering
\includegraphics[scale=0.4]{Schema/kevlar2.png} 
\vspace{-10pt}
\end{wrapfigure}

	Le Kevlar, aussi appelé \textit{poly(p-phénylène-téréphtalamide)} ou \textit{PPD-T}, est un polymère appartenant à la famille des fibres d'aramides\footnote{Mot condensé pour "aromatic polyamide"}. Sa formule chimique est \ce{[-CO-C_6H_4-CO-NH-C_6H_4-NH-]_n}. Il est synthétisé à partir de deux monomères :

\begin{itemize}
	\item la \textit{p-phénylènediamine}, \ce{C_6H_4(NH_2)_2}, un diamine aromatique utilisé principalement dans la synthèse des polymères et la production de teintures et colorants.
	\item le \textit{chlorure de téréphtaloyle}, \ce{C_8H_4Cl_2O_2}, entrant dans la synthèse des polymères et fibres d'aramides, leur permettant d'être très légers, solides, résistants au feu,... 
\end{itemize}

	%\begin{enumerate}
	%\item Synthétique : fait en laboratoire, ne vient pas de la nature.
	%\item Aromatique : possède une structure circulaire et très résistante, comme le benzène.
	%\item Polyamide : les structures circulaires se joignent entre elles pour former des chaînes très longues.
	%\item Polymère : le matériaux est constitué de molécules identiques liées entres elles.
	%\end{enumerate}

	%\begin{center}
	%	\includegraphics[scale=0.5]{Schema/formulechimique.png}  
	%\end{center}
	
	Ces deux monomères se lient les uns aux autres pour former des longues chaînes. Cependant, entre les deux monomères, il y a une disposition spéciale des molécules, les atomes d'oxygène et d'hydrogène se disposent alternativement de part et d'autres de la chaîne.  Cet agencement est nécessaire pour pouvoir former des liaisons hydrogènes entre les différentes chaînes polymères. Si toutes ces molécules étaient disposées du même côté de la chaîne, les liaisons n'auraient pas l'espace nécessaire pour se former. Au final, l'ensemble des longues chaînes de polymères parallèles reliées entre elles par des liaisons hydrogènes forment une structure planaire (semblable à la molécule de soie).

		
\subsection{Résistance à la traction}
	\begin{table}
	\centering
		\begin{tabular}{|c|p{4cm}|p{4cm}|p{4cm}|p{4cm}|}
		\hline

		Matériau & Résistance à la traction [\unit{\mega\pascal}] & Masse volumique [\unit{\gram\per\centi\cubic\meter}] & Résistance spécifique [\unit{\mega\pascal\usk\centi\cubic\meter\per\gram}]\\

		\hline
		Kevlar 49 & 3600-4100 & 1.44 & 2700\\
		\hline
		Fibre de carbone & 5700 & 1.5 & 3800 \\
		\hline
		Acier & 760 & 7.8 & 97 \\
		\hline
		Béton & 3 & 2.5 & 1.3 \\
		\hline
		\end{tabular}
	\caption{Propriétés mécaniques de différents matériaux}
	\label{tab:mecatab}
	\end{table}

Nous pouvons observer dans le tableau~\ref{tab:mecatab} qu’avec une résistance à la traction de plus de \unit{3600}{MPa} le kevlar possède une résistance cinq fois supérieure à celle de l’acier tout en étant plus léger. En effet, la masse volumique de kevlar est de $\unit{1.44}{g\per cm^3}$. Ce qui fait que la résistance spécifique\footnote{La résistance spécifique d’un matériau est sa résistance par rapport à sa masse volumique.} du kevlar est 28.7 fois supérieur à celle de l’acier. Nous constatons donc qu'à poids égal, le Kevlar est bien plus intéressant au niveau de la reprise d'efforts en traction que l'acier. Toutefois, le carbone supplante le Kevlar en traction, même si le carbone a une masse volumique légèrement plus importante que le Kevlar sa résistance en traction est bien plus élevée.


%La résistance à la traction du Kevlar vient surtout de sa composition interne.  
%Les liaisons hydrogènes sont formées entre 
%l'oxygène dense en électron et l'hydrogène déficient en électron.  Grâce au fait que les atomes d'oxygène et 
%d'hydrogène se situent de part et d'autre du groupe phényle, il y a des liaison hydrogènes formés à intervalles réguliers.  
%Par la résistance de ces liaisons et par leur dispersion très homogène, on sait maintenant pourquoi le Kevlar est très 
%résistant!  On peut également ajouter que les polymères de Kevlar ont une orientation radiale\footnote{La
%longueur de la fibre est parallèle au tissus}, et que c'est un matériau anisotrope\footnote{La résistance dans le sens de la fibre n'est pas la même que perpendiculairement à la fibre}.  A cause de cette géométrie structurelle, on comprend que la reprise en traction soit bien plus grande que la reprise en compression.

Pour expliquer pourquoi le Kevlar a une très bonne reprise en traction, il faut s'intéresser à la structure chimique du matériau, la force et la résistance du Kevlar venant surtout de sa composition interne.  Le Kevlar est formé de longues chaînes de molécules alignées parallèlement les unes par rapport aux autres procurant 
ainsi une forte résistance à la traction. Cette rigidité est due à la structure de base du kevlar. En effet, elle vient 
de la position en para\footnote{Désignent la position des substituants secondaires par rapport au substituant principal dans un cycle benzénique polysubstitué, Un substituant secondaire en position para sera sur l'atome opposé sur le cycle par rapport au substituant principal, c'est-à-dire en position « 4 ». L'isomère para est donc l'isomère 1,4. citer wiki} du noyau benzénique qui est le squelette de la molécule.  Une autre propriété du Kevlar qui 
contribue à sa résistance en traction sont les liaisons hydrogènes\footnote{La liaison hydrogène est une attraction 
dipôle-dipôle forte. Elle se présente lorsqu'on a un (ou plusieurs) atome d'hydrogène lié à un atome fortement
électromagnétique. C'est une liaison forte.} qui se passent entre l'hydrogène chargé positivement et le groupe 
carbonyle.  Dans le cas du Kevlar, les liaisons hydrogènes sont formées entre l'oxygène dense en électron et l'hydrogène déficient en électron.  Grâce au fait que les atomes d'oxygène et d'hydrogène se situe de part et d'autre du groupe phényle, il y a des liaisons hydrogènes formées à intervalles réguliers et cela permet d'obtenir une homogénéité dans la chaîne et donc d'avoir une résistance homogène sur le matériau.
\subsection{Résistance à la compression}
Un des défauts majeur du Kevlar est sa résistance en compression. En effet, le kevlar étant un matériaux anisotrope, celle-ci représente uniquement un dixième de sa résistance à la traction. La faible tenue mécanique en compression est généralement attribuée à une mauvaise adhérence des fibres à la matrice dans le matériau composite, on lui associe même un comportement ductile et non linéaire. Lors de la compression des fibres de Kevlar, celles-ci se plient et fléchissent comme le ferait un spaghetti sur lequel on pousse à chacune des ses extrémités, c'est le phénomène de flambement. Pour effectuer ces tests, il faut prendre une matrice de fibres soumise à la compression à ses extrémités. De plus, pour des contraintes allant de \unit{0.3}{\%} à \unit{0.5}{\%}, on remarque la formation de bande de plis, ce sont des défauts qui résultent du flambement que produit la compression. 
%\begin{figure}[h]
%\begin{center}
%\includegraphics[scale=0.3]{Schema/Kinkbands.png} 
%\end{center}
%\end{figure} 
%Manque de chimie (spaguetti) 
%A completer je fais ca demain !
% Sources : https://polycomp.mse.iastate.edu/files/2012/01/3-Aramid-Fibers.pdf


\subsection{Résistance à la chaleur}
%La résistance à la chaleur varie suivant l'épaisseur du fil pris en considération. Il se décompose à partir de \unit{450}{\celsius}. Une longue exposition à la température réduit la résistance à la traction, diminue son module d’élasticité et son élongation jusqu’à la rupture.  La température maximum recommandée pour une utilisation du kevlar dans l’air est de \unit{177}{\celsius}. Par contre, le kevlar offre une bonne résistance à l'enflammement.
%chimie
\begin{wrapfigure}{R}{0.16\textwidth}
 \vspace{-10pt}
\centering
\includegraphics[scale=0.55]{Schema/temp_tract.jpg} 
\vspace{-10pt}
\caption{Effets de la température sur la résistance à la traction}
\vspace{-5pt}
\label{temp_tract}
\end{wrapfigure} 
Il y a plusieurs impacts du à température sur les propriétés du Kevlar. Tout d'abord, il est important de noter que le diamètre des fibres ne varie pas après exposition à la chaleur. De plus, il se décompose à partir d'une température avoisinant les \unit{450}{\celsius}, tout dépendant bien sûr des conditions et de la durée d'exposition. Les figures~\ref{temp_tract} et~\ref{temp_young} montrent respectivement la diminution de la résistance à la traction et du module de Young à différentes températures et pour différentes durées d'exposition. On remarque que la résistance à la traction ne varie pas à \unit{100}{\celsius} et diminue d'approximativement de 20 et 30\% à respectivement \unit{200} et \unit{300}{\celsius}. Cette diminution en résistance est probablement le résultat d'un mécanisme d'oxydation provoquant une rupture dans la chaîne de polymère. De plus, il a déjà été observé que l'azote à plus de \unit{150}{\celsius} provoque une baisse de résistance de fibre, vraisemblablement par un mécanisme de nitruration. 

D'un autre côté, on peut noter, qu'en moyenne, il n'y a pas de baisse flagrante du module de Young, celui-ci n'est donc pas affecté par l'exposition à la chaleur.



\begin{wrapfigure}{L}{0.16\textwidth}
 \vspace{-10pt}
\centering
\includegraphics[scale=0.55]{Schema/temp_young.jpg} 
\vspace{-10pt}
\caption{Effets de la température sur le module de Young}
\vspace{-10pt}
\label{temp_young}
\end{wrapfigure} 

Le comportement des matériaux peut aussi être mesuré par le coefficient de dilatation thermique. Celui du Kevlar est négatif, comme la plupart des fibres. En effet, on observe que lorsque la température augmente, celles-ci subissent une diminution du volume. cependant, le coefficient du Kevlar est proche de 0, il n'y aura donc pas de variation notable du volume de la fibre avec la température.

%Le coefficient de dilatation thermique du kevlar est nul. Cela veut dire que quelle que soit la température, le volume du kevlar restera constant.
%
%\begin{wrapfigure}{R}{0.18\textwidth}
% \vspace{-20pt}
%\centering
%\includegraphics[scale=0.5]{Schema/dilatationterm.jpg} 
%\vspace{-10pt}
%\caption{Coefficient de dilatation thermique du Kevlar}
%\vspace{-5pt}
%\label{dilatationerm} 
%\end{wrapfigure} 
 

%On voit sur la figure~\ref{dilatationerm} que le coefficient de dilatation du Kevlar sur la partie longitudinale est négatif, ce qui veut dire qu'on a beau augmenter la température, le Kevlar ne se dilatera pas dans le sens de la longueur.



\begin{table}
\centering
\begin{tabular}{|c|p{3cm}|p{3cm}|p{3cm}|p{3cm}|}
\hline

Matériau & Module de Young [\unit{\giga\pascal}] & Limite d'élasticité [\unit{\mega\pascal}] & Coefficient de Poisson & Elongation à la rupture [\%] \\

\hline
Kevlar 49 &  112 & 3620 & 0.36 & 2.4 \\
\hline
Fibre de carbone &  220 & / & 0.3-0.35 & 1.4\\
\hline
Acier &  200 & 690 &0.3&  2\\
\hline
Béton & 20 & / &0.2& / \\
\hline
\end{tabular}
\caption{Comparaison de caractéristiques avec d'autres matériaux}
\vspace{-20pt}
\label{Atab}
\vspace{-3pt}
\end{table}


\subsection{Fluage}
Le fluage peut se décomposer en 3 étapes, la première phase se représente par un graphe de forme logarithmique avec le temps de charge.  Le fluage augmente donc beaucoup durant un petit pas de temps.  La deuxième étapes a une évolution linéaire par rapport au temps, durant cette étape, le fluage n'augmente pas beaucoup par rapport à la première étape,  c'est l'étape la plus longue des trois.   Enfin, la troisième étape est synonyme d'une grande déformation jusqu'à la rupture.  La vitese de déformation augmente avec le temps.  Il a été montré que le module de traction du Kevlar augmente avec la déformation en fluage.


\subsection{Resistance aux UV}
\begin{wrapfigure}{R}{0.16\textwidth}
 \vspace{-20pt}
\centering
\includegraphics[scale=0.3]{Schema/UVkevlar.jpg} 
\vspace{-10pt}
\caption{Effet des UV sur le kevlar}
\vspace{-20pt}
\label{UVkevlar}
\end{wrapfigure}
Comme toutes les matériaux polymères, le Kevlar est affecté par les UV's.  Les conséquences d'une exposition sont une perte de propriétés mécaniques et une décoloration de la fibre.   Nous pouvons en effet constater sur la figure~\ref{UVkevlar} qu’après plusieurs centaines d’heures d’expositions aux UV, le kevlar perd une bonne partie de sa résistance.  Cependant, une décoloration d'une fibre jeune n'est pas forcément lié à une dégradation de propriétés.  La dégradation dépend de la longueur d'onde, du temps d'exposition et de l'intensité de la lumière.   Il faut que 2 conditions soit réunis pour que les UV's dégradent une fibre: la fibre doit absorber les rayons ultraviolets et l'énergie de la lumière doit être assez grande que pour casser les liaisons chimiques.  Il faut surtout faire attention aux lumières de longueurs d'ondes comprises entre \unit {300} {nm} et \unit {500} {nm} car ce sont ces ondes que le Kevlar absorbe.  Les rayons du soleil contiennent ces longeurs d'onde et donc il faut protéger le Kevlar pour le maintenir performant.


 


\subsection{Résistance au PH}

Le Kevlar est virtuellement non affecté par un PH de 7, c’est-à-dire un PH neutre. Cependant, sa résistance diminue dans les milieux basiques et acides. Cependant, le Kevlar résiste mieux aux solutions basiques qu’aux solutions acides. A la figure~\ref{phkevlar} on voit l’évolution de la résistance du Kevlar plongé dans une solution pendant \unit{16}{h}.

%\begin{center}
%\includegraphics[scale=0.5]{phphoto.png} 
%\end{center}
\begin{wrapfigure}{L}{0.16\textwidth}
 \vspace{-20pt}
\centering
\includegraphics[scale=0.4]{Schema/phkevlar.jpg} 
\vspace{-10pt}
\caption{Effet du PH sur le kevlar \cite{DuPont}}
\vspace{-20pt}
\label{phkevlar}
\end{wrapfigure}


Riewald et al. ont réalisé une étude concernant la résistance des fibres Kevlar 29 et 49. Il en découle que les fibres perdent \unit{1.5}{\%} de leur résistance mécanique après un an d’immersion dans l’eau de mer.  Une autre étude montre qu'à PH9, la densité des fibres diminue fortement, c'est due à la coupure des chaines polymères ainsi qu'au developpement de la porosité.  A PH11, c'est uniquement due à la coupure des chaines polymères parce que la densité évolue peu dans ces conditions.




%\subsection{Qualités}
%Qu'en est-il des qualités du Kevlar ? Le Kevlar est utilisé dans une multitude d'applications pour ses propriétés de résistance, citons par exemple: les gilets pare-balles, dans le domaine automobile, aéronautique, ainsi que dans la structure des voiles de bateaux. Etant un matériaux cristallin, le Kelvlar est caractérisé par une rigidité et une résistance à la rupture exceptionnelle pour un polymère pour un poids relativement léger. De plus, il ne fond pas, mais se décompose juste pour des températures excédant les 400\degree à 450\degree. C'est un matériau fort et robuste qui à l'avantage d’être léger, il présente un bon module d'élasticité ainsi qu'un bon allongement sous charge. 
%
%\subsection{Défauts}
%Le Kevlar présente aussi des défauts, notamment des perturbations dues à l'humidité, c'est pourquoi il est généralement traîté en étuve. De plus, on rencontre des difficultés lorqu'il faut le couper ou encore le dimensionner. En effet, il présente une résistance à la coupure qui est de l'ordre de 5 fois celle du cuir.

%\section{Production}
%Le Kevlar est synthétisé à partir de monomères 1,4-phenyl-diamine et de terephthaloyl chloride.\\
%On le trouve a un prix élevé sur le marché à cause des difficultés dues à l'utilisation d'acide sulfurique concentré lors de sa production. 
%	


%\section{Sites}
%-http://www.ansellpro.com/auto/faq2.asp
%-http://www.dupont.com/products-and-services/fabrics-fibers-nonwovens/fibers/articles/kevlar-properties.html
%-http://www.explainthatstuff.com/kevlar.html



	
\begin{table}
\centering
\begin{tabular}{|c|p{3.1cm}|p{3.1cm}|p{3.1cm}|p{3.1cm}|}
\hline 
Conditions&Allongement à la rupture[\%]&Force à la rupture[GPa]&Module de Young[GPa]&Coefficient de poisson\\ 
\hline 
Non traité&$3.7\pm 0.2$&$3.66\pm 0.18 $&$93\pm 3$&$0.24$\\ 
\hline 
Traité dans l'eau&$3.6\pm 0.3$&$3.08\pm 0.35$&$80\pm 6$&$0.24$\\ 
\hline 
Traité aux UV&$3.2\pm 0.1$&$3.97\pm 0.26$&$120\pm 6$&$0.25$\\
\hline
\end{tabular} 
\caption{Propriétés mécaniques d'une seule fibre de kevlar T290}
\label{mecaniqueau}
\end{table}
\subsection{Résistance à l'eau}
Le groupe carbonyle ainsi que l'hydrogène sont présents dans chaque monomère de ce fait, les liaisons hydrogènes se font
de manière récurrente et multiple à travers les longues chaînes de molécules.

Cependant ces liaisons hydrogènes sont aussi une faiblesse du Kevlar, au contact de l'eau ces liaisons ont 
tendance à se casser pour privilégier des liaisons hydrogènes entre l'hydrogène du Kevlar et l'oxygène de l'eau définissant ainsi le comportement de reprise d'humidité du matériau. A la figure~\ref{hydrolyse}, on peux voir ce qu'il se passe structurellement lorsque l'eau brise les liaisons hydrogènes.
\begin{figure}
\centering
\includegraphics[scale=0.65]{Schema/hydrolyse.jpg} 
\caption{Mécanisme de l’hydrolyse du Kelvar \cite[p.~40]{Derombise2009comportement}}
\label{hydrolyse}
\end{figure}

Des recherches ont montré les effets de l'eau sur l'allongement à la rupture, la force à la rupture, le module de Young ainsi que le coefficient de poisson. Les résultats de ces recherches sont répertoriées dans le tableau~\ref{mecaniqueau}.


%site internet cool : http://www.christinedemerchant.com/carbon-kevlar-glass-comparison.html
 
 

\begin{wrapfigure}{L}{0.23\textwidth}
	\centering
	\vspace{-10pt}
	\begin{subfigure}[b]{0.13\textwidth}
	\centering
		\includegraphics[scale=0.3]{Schema/drykevlar.png}
		\caption{Kevlar sec}
        	\label{fig:drykev}
    \end{subfigure}%
    \begin{subfigure}[b]{0.13\textwidth}
    \centering
		\includegraphics[scale=0.3]{Schema/saturatedKevlar.png}
		\caption{Kevlar saturé}
        	\label{fig:satkev}
    \end{subfigure}
    \vspace{-20pt}
    \caption{Effet de l'eau sur le Kevlar}
    \vspace{-30pt}
    \label{waterkeveffect}
    
\end{wrapfigure}
On peut voir sur la figure~\ref{fig:satkev}, des fissures se prolongeant le long des fibres. Cette diffusion des fissures indique qu'elles ont été causées par la rupture des liaisons hydrogènes entre les monomères et par la fissuration de la matrice. Une telle détérioration rend compte de la diminution du module de Young. 



%\begin{figure}
%\begin{center}
%\begin{tabular}{|c|c|p{3cm}|}
%\hline
%Température $[\unit{}{\celsius}]$ & Temps $[\unit{}{jour}]$ & Dégradation de la résistance mécanique $[\unit{}{\%\per an}]$\\
%\hline
%200 & 2 & 12426\\
%\hline
%175 & 4 & 5585  \\
%\hline
%175 & 4 & 5466 \\
%\hline
%150 & 14 & 1563  \\
%\hline
%125 & 32 & 199\\
%\hline
%125 & 40 & 264  \\
%\hline
%100 & 162 & 64 \\
%\hline
%100 & 281 & 76-86  \\
%\hline
%\end{tabular}
%\caption{Taux de dégradation hydrolytique de la 
%résistance mécanique de fibres Kevlar 49 
%en fonction de la température à 100\% d'humidité relative.}
%\end{center}
%\end{figure}


%\begin{figure}[h]
%\centering
%\includegraphics[scale=0.6]{Schema/hydrolyse2.jpg} 
%\caption{Taux de dégradation hydrolytique de la 
%résistance mécanique de fibres Kevlar 49 
%en fonction de la température à 100\% d'humidité relative.}
%\label{hydrolyse2}
%\end{figure} 

%Geo's part
\section{Production}


La production du Kevlar comprend deux grandes étapes :

\begin{itemize}
	\item la polymérisation, à savoir la production du polymère à partir duquel le Kevlar est fait. Le polymère est synthétisé à partir de deux monomères : la \textit{p-phénylènediamine} ainsi que le \textit{chlorure de téréphtaloyle}. Cette étape est chimique.
	\item le filage du polymère, le transformer en une fibre solide, résistante. Cette phase est mécanique, il s'agit de manipuler le polymère afin de changer ses propriétés mécaniques.
\end{itemize}
	
	
\subsection{Polymérisation}

La réaction de la première étape est une polycondensation\footnote{Durant une réaction de condensation, deux molécules, ou groupements, se combinent pour former une molécule plus grosse tout en éliminant une plus petite molécule (le sous-produit, le plus souvent de l'eau). Chaque étape d'une polycondensation est une réaction de condensation.} réalisée à basse température en milieu solvant, qui aboutit à une longue chaîne de molécules, un polymère. La température de réaction se situe entre -15 et 30\celsius ~et la solution est continuellement remuée durant la polymérisation, pouvant aller de 2 à 24 heures. Le polyamide résultant est composé de noyaux aromatiques (benzène), alternés de groupements aromatiques. Initialement, le solvant était du hexamethylphosphoramide (HMPA) mais pour des raisons de toxicité (mineures), il a été remplacé par une solution de N-méthyl-2-pyrrolidone (NMP) et de chlorure de calcium.
\begin{figure}
\centering
\includegraphics[scale=0.215]{Schema/Kevlar_chemical_synthesis.png}
\caption{Synthétisation du kevlar}
\label{kevlarsynthesis}
\end{figure}
Le sous-produit de cette réaction est de l'acide chlorhydrique, \ce{HCl}. La présence de cet acide en solution pourrait provoquer des réactions de corrosion et endommager du matériel. Il est donc nécessaire d'introduire des bases (généralement des sels de lithium et carbone) pour neutraliser l'acidité.

Après cette étape, la solution à pris la forme d'une masse visqueuse semblable à du gel. Dans cette masse, il pourrait y avoir des sels insolubles en plus de la fibre voulue. Il faut donc les éliminer avant de continuer le processus.
		
\subsection{Filage}
	
A la sortie de la première étape, les fibres sont alignées aléatoirement. Cependant, la Kevlar à un comportement nématique, comme un cristal liquide. C'est-à-dire que les fibres ont tendance à s'aligner dans la même direction, par petits paquets, il y a un certain degré d'ordre naturel. En augmentant la concentration, le degré d'alignement augmente mais n'est pas encore parfait.

\begin{wrapfigure}{R}{0.18\textwidth}
 \vspace{-20pt}
\centering
\includegraphics[scale=0.6]{Schema/kevlar_1.jpg} 
\vspace{-10pt}
\caption{Filage}
\vspace{-20pt}
\label{filage} 
\end{wrapfigure} 


Pour remédier à celà vient la deuxième étape. Les fibres sont filées, lors d'un processus de filage humide. Les fibres sont dissoutes dans de l'acide sulfurique, \ce{H_2SO_4}, et la solution passe sous pression (pour augmenter la concentration) par de très petits orifices afin d'aligner le plus possible les fibres. Le dispositif est une filière. Lors du passage de la solution à travers les minuscules trous de la filière, les fibres s'alignent dans une même direction et le résultat est ensuite refroidi dans de l'eau froide. 

A la fin de cette étape, le résultat est un long fil de Kevlar. Les fils peuvent être utilisés comme tels, comme fil de pêche ou fil à coudre. Ils peuvent aussi être tressé pour former des câbles, des cordes. Une autre manière de les utiliser est de les tisser afin de créer des bandes pouvant entrer dans la composition de gilets pare-balles, gants résistants au feu,...


Sur la figure~\ref{Flowsheet}, on peut voir la flowsheet du procédé.
%	%Définir les formes
%	\tikzstyle{block} = [rectangle, draw, fill=blue!20,text width=14em, text badly centered, rounded corners, minimum height=4em, minimum width=23em, node distance=3cm] %Etape
%	\tikzstyle{line} = [draw, ->,>=latex] %Flèche
%	\tikzstyle{inn} = [ draw, ellipse,fill=green!60, minimum height=2em, align = center] %Entrée
%	\tikzstyle{outt} = [midway, draw, ellipse,fill=red!60, minimum height=2em, align = center] %Sortie
%	\tikzstyle{inout} = [midway, draw, ellipse,shade, top color = red!70, bottom color = green!70, minimum height=2em, align = center] %Entrée/Sortie
%	\tikzstyle{values} = [rectangle, draw, fill = gray!20] %Valeur
%	\tikzstyle{final} = [midway, diamond, draw, fill=yellow!60, text width=4.5em, text badly centered, node distance=3cm, inner sep=0pt] %Produit final


\begin{center}
\begin{wrapfigure}{L}{0.25\textwidth}
 \vspace{-20pt}
\centering

	\scalebox{0.6}{
	\begin{tikzpicture}
	    %Place nodes
	    
	    %Première étape : polymérisation 
	    \node[block] (Pol)at(0,13) {\textbf{Polymérisation} \\
	    \textit{Processus chimique}
	    $$\ce{[C_6H_4(NH_2)_2 + C_8H_4Cl_2O_2]_n} $$ 
	    $$\ce{->[\ce{-2 n HCl}]}$$
	    $$\ce{[-CO-C_6H_4-CO-NH-C_6H_4-NH-]_n} $$
	    };
	    
	    %Deuxième étape : filage
	    \node[block] (Fil)at(0,6) {\textbf{Filage} \\
	    \textit{Processus mécanique}
	    \ce{[-CO-C_6H_4-CO-NH-C_6H_4-NH-]_n} \\
	    $\downarrow$ \\
	    Kevlar
	    };
	
	    \node (in1)at(-5,19) {}; %Premier réactif -4
	    \node (in2)at(5,19) {};  %Deuxième réactif 4
	    \node (out1)at(10.5,13) {}; %Rejet de HCl 9
	    	\node (pfinal)at(0,0) {};  %Produit final
	    	\node[values] (outtt)at(7.2,14) {$306,723\kilogram = 8,404\kilo\mole$}; %Valeurs sur HCl 6.46 13.75
	    		    	
	    % Draw edges
	    
	    %Premier réactif
	    \path [line] (in1) -- node[inn] {p-phénylènediamine \\ \ce{C_6H_4(NH_2)_2}} node[values, near start] {$453,782\kilogram = 4,202\kilo\mole$} (Pol); 
	    %Deuxième réactif
	    \path [line] (in2) -- node[inn] {chlorure de téréphtaloyle \\ \ce{C_8H_4Cl_2O_2}} node[values, near start] {$852,941\kilogram = 4,202\kilo\mole$}(Pol); 
	    %Entre les deux étapes
	    \path [line] (Pol) -- node[inout] {PPD-T \\ \ce{[C_6H_4(NH_2)_2 + C_8H_4Cl_2O_2]_n}} node[values, near start] {$1000\kilogram = 4,202\kilo\mole$} (Fil); 
	    %Rejet de HCl
	    \path [line] (Pol) -- node[outt] {acide chlorhydrique \\ \ce{HCl}} (out1); 
	    %Produit final
	    \path [line] (Fil) -- node[final] {\textbf{Kevlar}} node[values, pos = 0.8] {$1000 \kilogram = 4,202\kilo\mole$} (pfinal);  

	\end{tikzpicture} }
	%\vspace{-5pt}
\caption{Flowsheet}
\vspace{-30pt}
\label{Flowsheet} 
\end{wrapfigure} 
\end{center}

\subsection{Coûts}

Le Kevlar est relativement cher à produire, bien qu'il soit difficile de trouver une estimation précise des coûts de production. Ce coût élevé est dû à différents paramètres. Tout d'abord, il y a derrière le Kevlar des années de recherches et d'innovation. De plus, la réaction nécessite des équipements bien spécifiques, et, par conséquent, peu répandus et chers. Enfin, une troisième cause du haut prix du Kevlar provient de l'étape de filage. En effet, le processus requiert l'utilisation d'acide sulfurique, qui, en plus d'être très coûteux, est extrêmement dangereux. Cela hausse le coût en ajoutant encore une contrainte au niveau des équipements nécessaires, qui doivent résister à cette substance et en exigeant des mesures de sécurité supplémentaires.

Le prix de vente du Kevlar est approximativement compris entre 20 et 40\euro ~du \unit{\cubic\meter}, voire plus, tout dépend de la qualité et du type de Kevlar.
\section{Application : voiles}

%\subsection{Introduction}
%   Le kevlar est un matériau utilisé dans une grande variété de domaines.  Parmi les applications les plus connues il y a les gilets pare-balles, des parties d’ailes d’avion, la coque des bateaux,... Comme application pour notre matériau nous avons décidé de nous intéresser à l’utilisation du Kevlar pour les voiles de voiliers et plus particulièrement les voiliers de compétition.  Grâce à ses propriétés physiques et chimiques incroyables, le Kevlar est toujours très largement utilisé dans les compétitions nautiques et cela, à juste titre.

\subsection{Le Kevlar, oui mais quel Kevlar?}
%\begin{wrapfigure}{L}{0.20\textwidth}
% \vspace{-10pt}
%\centering
%\includegraphics[scale=0.3]{Schema/kevlar_taffeta2.jpg} 
%\vspace{-10pt}
%\end{wrapfigure}
Tout d’abord, pourquoi se focaliser sur les voiliers de compétition et non les voiliers de plaisance ?  Simplement parce que le Kevlar présente des caractéristiques formidables pour la performance mais il présente des gros inconvénients pour des côtés plus pratiques et financiers. Penchons-nous donc plus en détails sur les propriétés de ce matériau. Nous nous focaliserons surtout sur le Kevlar 49 et le Kevlar 29. En effet, ces deux types de Kevlar sont principalement utilisés pour les voiles tandis que d’autres, tels que le 129, ne sauraient être utilisés pour faire une voile compétitive.

Le Kevlar 49 est 50\% plus résistant que le Kevlar 29 mais moins flexible. En ce qui concerne la voile, un bon équilibre entre la résistance et la flexibilité est nécessaire. Effectivement, une voile trop flexible va se déformer plus vite et donc apporter une résistance moindre au vent, entraînant une perte de propulsion du navire.  Un autre inconvénient d'une voile trop flexible est le fait qu'elle fasseye\footnote{Faseyer : En parlant d'une voile, flotter, battre au vent. Ce phénomène apparait lorsque la voile n'est pas assez tendue ou quand le vent n'est pas assez fort ou mal orienté. Il y a donc de la surface qui flotte comme du linge qui pend} et réduit la capacité totale de propulsion.  Au contraire, une voile trop peu flexible (Kevlar 129) va avoir une surface au vent moindre (car peu de déformation) et donc une surface de voile plus petite.  Un bon équilibre entre les deux va transformer plus efficacement la force du vent en vitesse. Il y a peu, la société DuPont a créé un nouveau Kevlar appelé le Kevlar Edge.  C’est un Kevlar hybride qui allie les points forts du Kevlar 29 et 49.  Le Kevlar Edge est 25\% plus flexible que le Kevlar 49 pour une résistance égale.  Cette nouvelle variante allie les avantages des deux modèles précédents. 

\subsection{Les avantages}
Le Kevlar a beaucoup de propriétés géniales pour la voile :  c’est un matériau extrêmement résistant (5 fois le module de l’acier à poids égal) et très léger.  Dans le domaine de la voile, il y a deux données principales à prendre en compte pour la vitesse, à savoir la propulsion et la traînée.  En allégeant au maximum le poids du voilier, la traînée sera réduite. En ayant une bonne résistance et flexibilité de la voile, la propulsion sera efficace.  Ensuite, le Kevlar a une très bonne résistance en traction. Ceci est un atout majeur sachant que le vent gonfle les voiles. Le Kevlar n’offrant pas de dilatation thermique notable, cela permet d’avoir une voile apportant les mêmes propriétés par tous les temps.  De plus, le fait que le Kevlar soit moins sensible en milieu alcalin qu’en milieu acide apporte un avantage énorme en compétition sur eau de mer. Puisque la mer est légèrement alcaline, la voile ne sera pas affectée par des éclaboussures ou autre mouillage, facteur primordial lorsqu’on passe plusieurs semaines en mer. 

\subsection{Les inconvénients}
Parmi les inconvénients du Kevlar il y a sa faible résistance aux rayonnements UV.  La couleur d’origine sable devient noir et la voile s’en trouve affectée. L’exposition aux UV, au soleil donc, affaiblit la souplesse et la solidité de la voile.  Celle-ci est aussi sensible au fluage, c’est-à-dire au pliage.  Le fluage endommage les voiles et il faut donc des véhicules adaptés pour les transporter, en raison de leur encombrement; même en les roulant, les voiles peuvent garder une taille assez conséquente.  On peut aussi citer une reprise d’humidité importante (4\%) qui va alourdir la voile quand celle-ci est mouillée et diminuer sa résistance.  L’usinage du Kevlar est assez complexe et donc le coût de production va s’en trouver augmenté.
\subsection{Alternative}
Une bonne alternative du Kevlar est le carbone. En effet, celui-ci présente une meilleure résistance en traction que le Kevlar et est donc tout autant utilisé dans le monde de la voile de compétition.  Il possède également un gros avantage par rapport au Kevlar : il est virtuellement insensbible aux UV, ce qui permet d'avoir une voile plus durable face au soleil. Les coûts d'entretien sont donc réduit à ce niveau-là. De plus, le coût du carbone est beaucoup plus faible que celui du Kevlar. Cependant un inconvénient du carbone est qu'il a tendance à moins se déformer que le Kevlar, engendrant une prise au vent plus faible. 


\subsection{Voiles avec plusieurs matériaux}
Le Kevlar peut aussi être utilisé pour renforcer une voile contenant déjà du Kevlar ou non : pour les coutures ou pour simplement amener plus de résistance à un endroit voulu. 
\subsubsection{Les voiles laminées}
Pour les voiles laminées, on utilise plusieurs matériaux afin de bénéficier des avantages de chacun d’eux.
La fibre (Kevlar pour notre cas) est prise en sandwich entre deux couches de film parfois appelé taffetas. La fibre procure une grande résistance à l'élongation (déformation de la voile), tandis que les deux films assurent la protection de la fibre en améliorant la résistance à l'abrasion et aux déchirures. Dans de récentes évolutions, des fils d'aramide sont inclus dans le film. On obtient donc un enchaînement film-fibre-film dont les fibres reprennent la plupart des efforts de traction alors que les films protègent essentiellement les fibres.  Cette technique offre une grande polyvalence pour la fabrication de voile.  On peut avoir une couche de tissu protectrice ou bien deux (pour alléger la voile ou le coût). 
\subsubsection{Fabrication en 3DL} Pour fabriquer ce type de voile, la technique utilisée est appelée 3DL.  Tout d’abord la voile est modélisée sur un ordinateur en 3D, ensuite, un programme lit le plan pour enfin envoyer les instructions à un robot qui fabrique sur mesure la base de la voile.  Des machines cousent les fils sur la base, donnant la première couche.  Il est primordial de mettre la même tension dans chaque fil pour que la voile ait une résistance uniforme.   L'étape suivante est un laminage (collage les fils et le film) pour bien fixer les deux couches.   Ensuite,la dernière couche est recouverte par un film qui sera par la suite tendu et fixé grâce à un grand sac sous vide qui va appliquer une très grande force pour sceller les différentes parties de la voile.  Pour finir, la voile est chauffée à une température bien spécifique pour fixer une dernière fois les différentes couches entre elles.
\subsubsection{Tissage}
Il est également possible de tisser une voile avec du fil de kevlar et de carbone.  Près des anneaux, il faut une très grande résistance pour que la voile  ne se déchire pas.  Le kevlar intervient donc et il y aura plus de tissage avec une grande quantité de kevlar dans cette partie-là.



%\end{multicols}
\end{document}