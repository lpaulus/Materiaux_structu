\section{Propriétés physiques et chimiques du matériaux}
\subsection{Disposition générale de la molécule}%c'est pas un peu pourri comme titre ? :p

%	Ces monomères se lient les uns aux autres pour former des longues chaînes. Entre le groupe amide et le groupe phényle, les molécules se disposent d'une manière spéciale, les \ce{O} et les \ce{H} se disposent de part et d'autres de la chaîne.  Cela est nécessaire pour pouvoir former des liaisons hydrogènes entre les différents monomères, si les \ce{C} et le \ce{H} étaient du même côtés de la chaîne, il y aurait trop peu d'espace pour que les liaisons hydrogènes puissent se former et dès lors la molécule perdrait beaucoup de résistance.
%	
%	 On peut également ajouter que les polymères de Kevlar ont une orientation radiale\footnote{c'est à dire que la longueur de la fibre est parallèle au tissus} et que c'est un matériau anisotrope\footnote{La résistance dans le sens de la fibre n'est pas la même que perpendiculairement à la fibre}.  A cause de cette géométrie structurelle, on comprend que la reprise en traction soit bien plus grande que la reprise en compression.  
	
\begin{wrapfigure}{L}{0.12\textwidth}
 \vspace{-10pt}
\centering
\includegraphics[scale=0.4]{Schema/kevlar2.png} 
\vspace{-10pt}
\end{wrapfigure}

	Le Kevlar, aussi appelé \textit{poly(p-phénylène-téréphtalamide)} ou \textit{PPD-T}, est un polymère appartenant à la famille des fibres d'aramides\footnote{Mot condensé pour "aromatic polyamide"}. Sa formule chimique est \ce{[-CO-C_6H_4-CO-NH-C_6H_4-NH-]_n}. Il est synthétisé à partir de deux monomères :

\begin{itemize}
	\item la \textit{p-phénylènediamine}, \ce{C_6H_4(NH_2)_2}, un diamine aromatique utilisé principalement dans la synthèse des polymères et la production de teintures et colorants.
	\item le \textit{chlorure de téréphtaloyle}, \ce{C_8H_4Cl_2O_2}, entrant dans la synthèse des polymères et fibres d'aramides, leur permettant d'être très légers, solides, résistants au feu,... 
\end{itemize}

	%\begin{enumerate}
	%\item Synthétique : fait en laboratoire, ne vient pas de la nature.
	%\item Aromatique : possède une structure circulaire et très résistante, comme le benzène.
	%\item Polyamide : les structures circulaires se joignent entre elles pour former des chaînes très longues.
	%\item Polymère : le matériaux est constitué de molécules identiques liées entres elles.
	%\end{enumerate}

	%\begin{center}
	%	\includegraphics[scale=0.5]{Schema/formulechimique.png}  
	%\end{center}
	
	Ces deux monomères se lient les uns aux autres pour former des longues chaînes. Cependant, entre les deux monomères, il y a une disposition spéciale des molécules, les atomes d'oxygène et d'hydrogène se disposent alternativement de part et d'autres de la chaîne.  Cet agencement est nécessaire pour pouvoir former des liaisons hydrogènes entre les différentes chaînes polymères. Si toutes ces molécules étaient disposées du même côté de la chaîne, les liaisons n'auraient pas l'espace nécessaire pour se former. Au final, l'ensemble des longues chaînes de polymères parallèles reliées entre elles par des liaisons hydrogènes forment une structure planaire (semblable à la molécule de soie).

		
\subsection{Résistance à la traction}
	\begin{table}
	\centering
		\begin{tabular}{|c|p{4cm}|p{4cm}|p{4cm}|p{4cm}|}
		\hline

		Matériau & Résistance à la traction [\unit{\mega\pascal}] & Masse volumique [\unit{\gram\per\centi\cubic\meter}] & Résistance spécifique [\unit{\mega\pascal\usk\centi\cubic\meter\per\gram}]\\

		\hline
		Kevlar 49 & 3600-4100 & 1.44 & 2700\\
		\hline
		Fibre de carbone & 5700 & 1.5 & 3800 \\
		\hline
		Acier & 760 & 7.8 & 97 \\
		\hline
		Béton & 3 & 2.5 & 1.3 \\
		\hline
		\end{tabular}
	\caption{Propriétés mécaniques de différents matériaux}
	\label{tab:mecatab}
	\end{table}

Nous pouvons observer dans le tableau~\ref{tab:mecatab} qu’avec une résistance à la traction de plus de \unit{3600}{MPa} le kevlar possède une résistance cinq fois supérieure à celle de l’acier tout en étant plus léger. En effet, la masse volumique de kevlar est de $\unit{1.44}{g\per cm^3}$. Ce qui fait que la résistance spécifique\footnote{La résistance spécifique d’un matériau est sa résistance par rapport à sa masse volumique.} du kevlar est 28.7 fois supérieur à celle de l’acier. Nous constatons donc qu'à poids égal, le Kevlar est bien plus intéressant au niveau de la reprise d'efforts en traction que l'acier. Toutefois, le carbone supplante le Kevlar en traction, même si le carbone a une masse volumique légèrement plus importante que le Kevlar sa résistance en traction est bien plus élevée.


%La résistance à la traction du Kevlar vient surtout de sa composition interne.  
%Les liaisons hydrogènes sont formées entre 
%l'oxygène dense en électron et l'hydrogène déficient en électron.  Grâce au fait que les atomes d'oxygène et 
%d'hydrogène se situent de part et d'autre du groupe phényle, il y a des liaison hydrogènes formés à intervalles réguliers.  
%Par la résistance de ces liaisons et par leur dispersion très homogène, on sait maintenant pourquoi le Kevlar est très 
%résistant!  On peut également ajouter que les polymères de Kevlar ont une orientation radiale\footnote{La
%longueur de la fibre est parallèle au tissus}, et que c'est un matériau anisotrope\footnote{La résistance dans le sens de la fibre n'est pas la même que perpendiculairement à la fibre}.  A cause de cette géométrie structurelle, on comprend que la reprise en traction soit bien plus grande que la reprise en compression.

Pour expliquer pourquoi le Kevlar a une très bonne reprise en traction, il faut s'intéresser à la structure chimique du matériau, la force et la résistance du Kevlar venant surtout de sa composition interne.  Le Kevlar est formé de longues chaînes de molécules alignées parallèlement les unes par rapport aux autres procurant 
ainsi une forte résistance à la traction. Cette rigidité est due à la structure de base du kevlar. En effet, elle vient 
de la position en para\footnote{Désignent la position des substituants secondaires par rapport au substituant principal dans un cycle benzénique polysubstitué, Un substituant secondaire en position para sera sur l'atome opposé sur le cycle par rapport au substituant principal, c'est-à-dire en position « 4 ». L'isomère para est donc l'isomère 1,4. citer wiki} du noyau benzénique qui est le squelette de la molécule.  Une autre propriété du Kevlar qui 
contribue à sa résistance en traction sont les liaisons hydrogènes\footnote{La liaison hydrogène est une attraction 
dipôle-dipôle forte. Elle se présente lorsqu'on a un (ou plusieurs) atome d'hydrogène lié à un atome fortement
électromagnétique. C'est une liaison forte.} qui se passent entre l'hydrogène chargé positivement et le groupe 
carbonyle.  Dans le cas du Kevlar, les liaisons hydrogènes sont formées entre l'oxygène dense en électron et l'hydrogène déficient en électron.  Grâce au fait que les atomes d'oxygène et d'hydrogène se situe de part et d'autre du groupe phényle, il y a des liaisons hydrogènes formées à intervalles réguliers et cela permet d'obtenir une homogénéité dans la chaîne et donc d'avoir une résistance homogène sur le matériau.
\subsection{Résistance à la compression}
Un des défauts majeur du Kevlar est sa résistance en compression. En effet, le kevlar étant un matériaux anisotrope, celle-ci représente uniquement un dixième de sa résistance à la traction. La faible tenue mécanique en compression est généralement attribuée à une mauvaise adhérence des fibres à la matrice dans le matériau composite, on lui associe même un comportement ductile et non linéaire. Lors de la compression des fibres de Kevlar, celles-ci se plient et fléchissent comme le ferait un spaghetti sur lequel on pousse à chacune des ses extrémités, c'est le phénomène de flambement. Pour effectuer ces tests, il faut prendre une matrice de fibres soumise à la compression à ses extrémités. De plus, pour des contraintes allant de \unit{0.3}{\%} à \unit{0.5}{\%}, on remarque la formation de bande de plis, ce sont des défauts qui résultent du flambement que produit la compression. 
%\begin{figure}[h]
%\begin{center}
%\includegraphics[scale=0.3]{Schema/Kinkbands.png} 
%\end{center}
%\end{figure} 
%Manque de chimie (spaguetti) 
%A completer je fais ca demain !
% Sources : https://polycomp.mse.iastate.edu/files/2012/01/3-Aramid-Fibers.pdf


\subsection{Résistance à la chaleur}
%La résistance à la chaleur varie suivant l'épaisseur du fil pris en considération. Il se décompose à partir de \unit{450}{\celsius}. Une longue exposition à la température réduit la résistance à la traction, diminue son module d’élasticité et son élongation jusqu’à la rupture.  La température maximum recommandée pour une utilisation du kevlar dans l’air est de \unit{177}{\celsius}. Par contre, le kevlar offre une bonne résistance à l'enflammement.
%chimie
\begin{wrapfigure}{R}{0.16\textwidth}
 \vspace{-10pt}
\centering
\includegraphics[scale=0.55]{Schema/temp_tract.jpg} 
\vspace{-10pt}
\caption{Effets de la température sur la résistance à la traction}
\vspace{-5pt}
\label{temp_tract}
\end{wrapfigure} 
Il y a plusieurs impacts du à température sur les propriétés du Kevlar. Tout d'abord, il est important de noter que le diamètre des fibres ne varie pas après exposition à la chaleur. De plus, il se décompose à partir d'une température avoisinant les \unit{450}{\celsius}, tout dépendant bien sûr des conditions et de la durée d'exposition. Les figures~\ref{temp_tract} et~\ref{temp_young} montrent respectivement la diminution de la résistance à la traction et du module de Young à différentes températures et pour différentes durées d'exposition. On remarque que la résistance à la traction ne varie pas à \unit{100}{\celsius} et diminue d'approximativement de 20 et 30\% à respectivement \unit{200} et \unit{300}{\celsius}. Cette diminution en résistance est probablement le résultat d'un mécanisme d'oxydation provoquant une rupture dans la chaîne de polymère. De plus, il a déjà été observé que l'azote à plus de \unit{150}{\celsius} provoque une baisse de résistance de fibre, vraisemblablement par un mécanisme de nitruration. 

D'un autre côté, on peut noter, qu'en moyenne, il n'y a pas de baisse flagrante du module de Young, celui-ci n'est donc pas affecté par l'exposition à la chaleur.



\begin{wrapfigure}{L}{0.16\textwidth}
 \vspace{-10pt}
\centering
\includegraphics[scale=0.55]{Schema/temp_young.jpg} 
\vspace{-10pt}
\caption{Effets de la température sur le module de Young}
\vspace{-10pt}
\label{temp_young}
\end{wrapfigure} 

Le comportement des matériaux peut aussi être mesuré par le coefficient de dilatation thermique. Celui du Kevlar est négatif, comme la plupart des fibres. En effet, on observe que lorsque la température augmente, celles-ci subissent une diminution du volume. cependant, le coefficient du Kevlar est proche de 0, il n'y aura donc pas de variation notable du volume de la fibre avec la température.

%Le coefficient de dilatation thermique du kevlar est nul. Cela veut dire que quelle que soit la température, le volume du kevlar restera constant.
%
%\begin{wrapfigure}{R}{0.18\textwidth}
% \vspace{-20pt}
%\centering
%\includegraphics[scale=0.5]{Schema/dilatationterm.jpg} 
%\vspace{-10pt}
%\caption{Coefficient de dilatation thermique du Kevlar}
%\vspace{-5pt}
%\label{dilatationerm} 
%\end{wrapfigure} 
 

%On voit sur la figure~\ref{dilatationerm} que le coefficient de dilatation du Kevlar sur la partie longitudinale est négatif, ce qui veut dire qu'on a beau augmenter la température, le Kevlar ne se dilatera pas dans le sens de la longueur.



\begin{table}
\centering
\begin{tabular}{|c|p{3cm}|p{3cm}|p{3cm}|p{3cm}|}
\hline

Matériau & Module de Young [\unit{\giga\pascal}] & Limite d'élasticité [\unit{\mega\pascal}] & Coefficient de Poisson & Elongation à la rupture [\%] \\

\hline
Kevlar 49 &  112 & 3620 & 0.36 & 2.4 \\
\hline
Fibre de carbone &  220 & / & 0.3-0.35 & 1.4\\
\hline
Acier &  200 & 690 &0.3&  2\\
\hline
Béton & 20 & / &0.2& / \\
\hline
\end{tabular}
\caption{Comparaison de caractéristiques avec d'autres matériaux}
\vspace{-20pt}
\label{Atab}
\vspace{-3pt}
\end{table}


\subsection{Fluage}
Le fluage peut se décomposer en 3 étapes, la première phase se représente par un graphe de forme logarithmique avec le temps de charge.  Le fluage augmente donc beaucoup durant un petit pas de temps.  La deuxième étapes a une évolution linéaire par rapport au temps, durant cette étape, le fluage n'augmente pas beaucoup par rapport à la première étape,  c'est l'étape la plus longue des trois.   Enfin, la troisième étape est synonyme d'une grande déformation jusqu'à la rupture.  La vitese de déformation augmente avec le temps.  Il a été montré que le module de traction du Kevlar augmente avec la déformation en fluage.


\subsection{Resistance aux UV}
\begin{wrapfigure}{R}{0.16\textwidth}
 \vspace{-20pt}
\centering
\includegraphics[scale=0.3]{Schema/UVkevlar.jpg} 
\vspace{-10pt}
\caption{Effet des UV sur le kevlar}
\vspace{-20pt}
\label{UVkevlar}
\end{wrapfigure}
Comme toutes les matériaux polymères, le Kevlar est affecté par les UV's.  Les conséquences d'une exposition sont une perte de propriétés mécaniques et une décoloration de la fibre.   Nous pouvons en effet constater sur la figure~\ref{UVkevlar} qu’après plusieurs centaines d’heures d’expositions aux UV, le kevlar perd une bonne partie de sa résistance.  Cependant, une décoloration d'une fibre jeune n'est pas forcément lié à une dégradation de propriétés.  La dégradation dépend de la longueur d'onde, du temps d'exposition et de l'intensité de la lumière.   Il faut que 2 conditions soit réunis pour que les UV's dégradent une fibre: la fibre doit absorber les rayons ultraviolets et l'énergie de la lumière doit être assez grande que pour casser les liaisons chimiques.  Il faut surtout faire attention aux lumières de longueurs d'ondes comprises entre \unit {300} {nm} et \unit {500} {nm} car ce sont ces ondes que le Kevlar absorbe.  Les rayons du soleil contiennent ces longeurs d'onde et donc il faut protéger le Kevlar pour le maintenir performant.


 


\subsection{Résistance au PH}

Le Kevlar est virtuellement non affecté par un PH de 7, c’est-à-dire un PH neutre. Cependant, sa résistance diminue dans les milieux basiques et acides. Cependant, le Kevlar résiste mieux aux solutions basiques qu’aux solutions acides. A la figure~\ref{phkevlar} on voit l’évolution de la résistance du Kevlar plongé dans une solution pendant \unit{16}{h}.

%\begin{center}
%\includegraphics[scale=0.5]{phphoto.png} 
%\end{center}
\begin{wrapfigure}{L}{0.16\textwidth}
 \vspace{-20pt}
\centering
\includegraphics[scale=0.4]{Schema/phkevlar.jpg} 
\vspace{-10pt}
\caption{Effet du PH sur le kevlar \cite{DuPont}}
\vspace{-20pt}
\label{phkevlar}
\end{wrapfigure}


Riewald et al. ont réalisé une étude concernant la résistance des fibres Kevlar 29 et 49. Il en découle que les fibres perdent \unit{1.5}{\%} de leur résistance mécanique après un an d’immersion dans l’eau de mer.  Une autre étude montre qu'à PH9, la densité des fibres diminue fortement, c'est due à la coupure des chaines polymères ainsi qu'au developpement de la porosité.  A PH11, c'est uniquement due à la coupure des chaines polymères parce que la densité évolue peu dans ces conditions.




%\subsection{Qualités}
%Qu'en est-il des qualités du Kevlar ? Le Kevlar est utilisé dans une multitude d'applications pour ses propriétés de résistance, citons par exemple: les gilets pare-balles, dans le domaine automobile, aéronautique, ainsi que dans la structure des voiles de bateaux. Etant un matériaux cristallin, le Kelvlar est caractérisé par une rigidité et une résistance à la rupture exceptionnelle pour un polymère pour un poids relativement léger. De plus, il ne fond pas, mais se décompose juste pour des températures excédant les 400\degree à 450\degree. C'est un matériau fort et robuste qui à l'avantage d’être léger, il présente un bon module d'élasticité ainsi qu'un bon allongement sous charge. 
%
%\subsection{Défauts}
%Le Kevlar présente aussi des défauts, notamment des perturbations dues à l'humidité, c'est pourquoi il est généralement traîté en étuve. De plus, on rencontre des difficultés lorqu'il faut le couper ou encore le dimensionner. En effet, il présente une résistance à la coupure qui est de l'ordre de 5 fois celle du cuir.

%\section{Production}
%Le Kevlar est synthétisé à partir de monomères 1,4-phenyl-diamine et de terephthaloyl chloride.\\
%On le trouve a un prix élevé sur le marché à cause des difficultés dues à l'utilisation d'acide sulfurique concentré lors de sa production. 
%	


%\section{Sites}
%-http://www.ansellpro.com/auto/faq2.asp
%-http://www.dupont.com/products-and-services/fabrics-fibers-nonwovens/fibers/articles/kevlar-properties.html
%-http://www.explainthatstuff.com/kevlar.html



	
\begin{table}
\centering
\begin{tabular}{|c|p{3.1cm}|p{3.1cm}|p{3.1cm}|p{3.1cm}|}
\hline 
Conditions&Allongement à la rupture[\%]&Force à la rupture[GPa]&Module de Young[GPa]&Coefficient de poisson\\ 
\hline 
Non traité&$3.7\pm 0.2$&$3.66\pm 0.18 $&$93\pm 3$&$0.24$\\ 
\hline 
Traité dans l'eau&$3.6\pm 0.3$&$3.08\pm 0.35$&$80\pm 6$&$0.24$\\ 
\hline 
Traité aux UV&$3.2\pm 0.1$&$3.97\pm 0.26$&$120\pm 6$&$0.25$\\
\hline
\end{tabular} 
\caption{Propriétés mécaniques d'une seule fibre de kevlar T290}
\label{mecaniqueau}
\end{table}
\subsection{Résistance à l'eau}
Le groupe carbonyle ainsi que l'hydrogène sont présents dans chaque monomère de ce fait, les liaisons hydrogènes se font
de manière récurrente et multiple à travers les longues chaînes de molécules.

Cependant ces liaisons hydrogènes sont aussi une faiblesse du Kevlar, au contact de l'eau ces liaisons ont 
tendance à se casser pour privilégier des liaisons hydrogènes entre l'hydrogène du Kevlar et l'oxygène de l'eau définissant ainsi le comportement de reprise d'humidité du matériau. A la figure~\ref{hydrolyse}, on peux voir ce qu'il se passe structurellement lorsque l'eau brise les liaisons hydrogènes.
\begin{figure}
\centering
\includegraphics[scale=0.65]{Schema/hydrolyse.jpg} 
\caption{Mécanisme de l’hydrolyse du Kelvar \cite[p.~40]{Derombise2009comportement}}
\label{hydrolyse}
\end{figure}

Des recherches ont montré les effets de l'eau sur l'allongement à la rupture, la force à la rupture, le module de Young ainsi que le coefficient de poisson. Les résultats de ces recherches sont répertoriées dans le tableau~\ref{mecaniqueau}.


%site internet cool : http://www.christinedemerchant.com/carbon-kevlar-glass-comparison.html
 
 

\begin{wrapfigure}{L}{0.23\textwidth}
	\centering
	\vspace{-10pt}
	\begin{subfigure}[b]{0.13\textwidth}
	\centering
		\includegraphics[scale=0.3]{Schema/drykevlar.png}
		\caption{Kevlar sec}
        	\label{fig:drykev}
    \end{subfigure}%
    \begin{subfigure}[b]{0.13\textwidth}
    \centering
		\includegraphics[scale=0.3]{Schema/saturatedKevlar.png}
		\caption{Kevlar saturé}
        	\label{fig:satkev}
    \end{subfigure}
    \vspace{-20pt}
    \caption{Effet de l'eau sur le Kevlar}
    \vspace{-30pt}
    \label{waterkeveffect}
    
\end{wrapfigure}
On peut voir sur la figure~\ref{fig:satkev}, des fissures se prolongeant le long des fibres. Cette diffusion des fissures indique qu'elles ont été causées par la rupture des liaisons hydrogènes entre les monomères et par la fissuration de la matrice. Une telle détérioration rend compte de la diminution du module de Young. 



%\begin{figure}
%\begin{center}
%\begin{tabular}{|c|c|p{3cm}|}
%\hline
%Température $[\unit{}{\celsius}]$ & Temps $[\unit{}{jour}]$ & Dégradation de la résistance mécanique $[\unit{}{\%\per an}]$\\
%\hline
%200 & 2 & 12426\\
%\hline
%175 & 4 & 5585  \\
%\hline
%175 & 4 & 5466 \\
%\hline
%150 & 14 & 1563  \\
%\hline
%125 & 32 & 199\\
%\hline
%125 & 40 & 264  \\
%\hline
%100 & 162 & 64 \\
%\hline
%100 & 281 & 76-86  \\
%\hline
%\end{tabular}
%\caption{Taux de dégradation hydrolytique de la 
%résistance mécanique de fibres Kevlar 49 
%en fonction de la température à 100\% d'humidité relative.}
%\end{center}
%\end{figure}


%\begin{figure}[h]
%\centering
%\includegraphics[scale=0.6]{Schema/hydrolyse2.jpg} 
%\caption{Taux de dégradation hydrolytique de la 
%résistance mécanique de fibres Kevlar 49 
%en fonction de la température à 100\% d'humidité relative.}
%\label{hydrolyse2}
%\end{figure} 
