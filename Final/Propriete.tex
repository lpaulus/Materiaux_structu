\section{Propriétés physiques et chimiques du matériaux}


\begin{figure}
\begin{center}
\begin{tabular}{|c|c|c|c|c|}
\hline

matériau & résistance à la traction $[MPa]$ & masse volumique $[\frac{g}{cm^3}]$& résistance spécifique $[\frac{MPa \cdot cm^3}{g}]$\\

\hline
Kelvar 49 & 3600-4100 & 1.44 & 2700\\
\hline
Carbone & 5700 & 1.5 & 3800 \\
\hline
Acier & 760 & 7.8 & 97 \\
\hline
Béton & 3 & 2.5 & 1.3 \\
\hline
\end{tabular}
\end{center}
\end{figure}

\paragraph{}
Nous pouvons observer dans ce tableau qu’avec une résistance à la traction de plus de 3600 MPa le kevlar possède une résistance cinq fois supérieure à celle de l’acier en étant plus léger. En effet, la masse volumique de kevlar est d’1.44 g/cm3. Ce qui fait la résistance spécifique du kevlar est 28.7 fois supérieur à celui de l’acier. La résistance spécifique d’un matériau est sa résistance par rapport à sa masse volumique. Par contre, le kevlar possède une résistance spécifique inférieure à celle du carbone. Nous savons aussi qu'avec une ductilité de 3\%, le kevlar n'est pas un matériau des plus ductile.

\paragraph{}
 Le kevlar est un matériau anisotrope. Cela veut dire que sa résistance dans le sens de la fibre n’est pas la même que perpendiculairement à la fibre. C’est à cause de cette anisotropie que le kevlar n'est généralement pas utilisé comme matériau seul mais comme renfort.
%plus détaillé
\paragraph{}
Le coefficient de dilatation thermique du kevlar est nul. Cela veut dire que quelle que soit la température, le volume du kevlar restera constant.  En plus de cela il est insensible aux produits chimiques mécaniques comme les graisses, huiles, solvants et au pétrole.
\begin{figure}
\begin{center}
\begin{tabular}{|c|c|c|c|}
\hline

matériau & ténacité $[MPa]$ & module $[GPa]$ & élongation à la rupture \\

\hline
Kelvar 29 & 2920 & 70 & 3.6 \\
\hline
Kelvar 49 & 3000 & 112 & 2.4 \\
\hline
Carbone & 3100 & 220 & 1.4\\
\hline
Acier & 1965 & 200 &  2\\
\hline
\end{tabular}
\end{center}
\end{figure}
\paragraph{}
Comme nous pouvons le constater, le kevlar possède un module d’élasticité de minimum \unit{70}{GPa}. Même si cette fois il est inférieur au module de l’acier c’est quand même une bonne valeur.  Le kevlar a également une haute ténacité. La ténacité caractérise le comportement d’un matériau à la rupture en présence d’une entaille.

\paragraph{Résistance à la compression}
Par contre le kevlar a une très mauvaise résistance à la compression. La résistance à la compression du kevlar est égale à un dixième de sa résistance à la traction. La faible tenue mécanique en compression est généralement attribuée à une mauvaise adhérence des fibres à la matrice dans le matériau composite
%Manque de chimie (spaguetti)

\paragraph{Resistance aux UV}
L’un de ses principaux défauts est sa résistance aux UV. Nous pouvons en effet constater via le graphe ci-dessous qu’après plusieurs centaines d’heures d’expositions aux UV, le kevlar perd une bonne partie de sa résistance.


%\begin{figure}[h]
%\begin{center}
%\includegraphics[scale=0.3]{Schema/UVkevlar.png} 
%\end{center}
%\end{figure} 

\paragraph{La chaleur}
Un autre défaut du kevlar est son comportement face à la chaleur. La résistance à la chaleur varie suivant l'épaisseur du fil pris en considération. Il se décompose à partir de \unit{450}{\celsius}. Une longue exposition à la température réduit la résistance à la traction, diminue son module d’élasticité et son élongation jusqu’à la rupture. La température maximum recommandée pour une utilisation du kevlar dans l’air est de \unit{177}{\celsius}. Le kevlar offre en plus une bonne résistance à l'enflammement.
%chimie

%\begin{center}
%\includegraphics[scale=1]{Schema/temperature.png} 
%\end{center}
%
%
%\begin{center}
%\includegraphics[scale=0.3]{Schema/temperature3.png} 
%\end{center}

\paragraph{La reprise d'humidité}
Un autre défaut du kevlar est sa grande reprise d'humidité. Le kevlar absorbe rapidement l'humidité présente dans l'air. Le kevlar est sensibles à l'hydrolyse suite à la présence de fonctions amides. Le schéma ci-dessous montre la réaction d'une fonction amide en présence d'eau.

%\begin{figure}[h]
%\begin{center}
%\includegraphics[scale=0.65]{Schema/hydrolyse.png} 
%\caption{Mécanisme de l’hydrolyse du PPTA}
%\end{center}
%\end{figure} 


Pour expliquer le fait que le kevlar ait une grande reprise d'humidité nous avons analysé la composition chimique du kevlar en contact avec de l'eau. Ce mécanisme casse des chaînes au niveau des liaisons C-N de la fonction amide. Ces ruptures impliquent la formation de chaînes acide et amine. Il a été prouvé que ces ruptures de chaînes induisent une diminution de la résistance mécanique du kevlar.


%\begin{figure}[h]
%\begin{center}
%\includegraphics[scale=0.6]{Schema/hydrolyse2.png} 
%\caption{Taux de dégradation hydrolytique de la 
%résistance mécanique de fibres Kevlar 49 
%en fonction de la température à 100\% d'humidité relative. }
%\end{center}
%\end{figure} 

\paragraph{Résistance au PH}

Le Kevlar est virtuellement non affecté par un PH de 7, c’est-à-dire un PH neutre.  Cependant, sa résistance diminue dans les milieux basiques et acides.  La résistance décroit au fur et à mesure qu’on s’éloigne du PH neutre.  Mais le Kevlar résiste mieux aux solutions basiques qu’aux solutions acides.   Voilà un schéma qui montre l’évolution de la résistance du Kevlar plongé dans une solution pendant 16h.

%\begin{center}
%\includegraphics[scale=0.5]{phphoto.png} 
%\end{center}

Riewald et al. a réalisé une étude concernant la résistance des fibres Kevlar 29 et 49 et il en découle que les fibres perdent 1.5\% de leur résistance mécanique après un an d’immersion dans l’eau de mer.



\section{Propriété physique} 
Il est constitué d'une longue chaine de polymères orientés de manière parallèle. Il en existe de divers grade, citons par exemple le Kevlar 29, Kevlar 49, Kevlar 129... 

%\subsection{Qualités}
%Qu'en est-il des qualités du Kevlar ? Le Kevlar est utilisé dans une multitude d'applications pour ses propriétés de résistance, citons par exemple: les gilets pare-balles, dans le domaine automobile, aéronautique, ainsi que dans la structure des voiles de bateaux. Etant un matériaux cristallin, le Kelvlar est caractérisé par une rigidité et une résistance à la rupture exceptionnelle pour un polymère pour un poids relativement léger. De plus, il ne fond pas, mais se décompose juste pour des températures excédant les 400\degree à 450\degree. C'est un matériau fort et robuste qui à l'avantage d’être léger, il présente un bon module d'élasticité ainsi qu'un bon allongement sous charge. 
%
%\subsection{Défauts}
%Le Kevlar présente aussi des défauts, notamment des perturbations dues à l'humidité, c'est pourquoi il est généralement traîté en étuve. De plus, on rencontre des difficultés lorqu'il faut le couper ou encore le dimensionner. En effet, il présente une résistance à la coupure qui est de l'ordre de 5 fois celle du cuir.

%\section{Production}
%Le Kevlar est synthétisé à partir de monomères 1,4-phenyl-diamine et de terephthaloyl chloride.\\
%On le trouve a un prix élevé sur le marché à cause des difficultés dues à l'utilisation d'acide sulfurique concentré lors de sa production. 
%	
\section{Heat resistance}
Qu'en est-il de la résistance à la chaleur de ce matériau ? 
La résistance à la chaleur varie suivant l'épaisseur du fil pris en considération. Pour du kevlar de grosse épaisseur, il peut résister jusqu'à des températures atteignant les 200\degree. Le transfert de chaleur au sein du kevlar se déroule plus lentement que losrqu'on parle de coton ou encore du cuir. Cela peut s'avérer interressant dans des utilisations qui nécessite par exemple de transporter des objets ou autre à haute température.

Cependant, le coton est toujours utilisé pour une gamme moyenne de température car il dissipe mieux la chaleur et est également une fibre naturelle qui "respire" mieux que le kevlar. En matière de résistance à la chaleur, on peut dire que le kevlar et le coton se valent, cependant le kevlar offre en plus une meilleure résistance à l'enflammement.

\section{Sites}
-http://www.ansellpro.com/auto/faq2.asp
-http://www.dupont.com/products-and-services/fabrics-fibers-nonwovens/fibers/articles/kevlar-properties.html
-http://www.explainthatstuff.com/kevlar.html




\chapter{Structure chimique}
\section{}

Le Kevlar est formé de longues chaînes de molécules alignées parallèlement les unes par rapport aux autres procurant 
ainsi une forte résistance à la traction. Cette rigidité est due à la structure de base du kevlar, en effet elle vient 
de la position en para du noyau benzénique qui est le squelette de la molécule.  Une autre propriété du Kevlar qui 
contribue à sa résistance en traction sont les liaisons hydrogènes\footnote{La liaison hydrogène est une attraction 
dipôle-dipôle forte. Elle se présente lorsqu'on a un (ou plusieurs) atome d'hydrogène lié à un atome fortement
électromagnétique. C'est une liaison forte.} qui se passent entre l'hydrogène chargé positivement et le groupe 
carbonyle.

	Le Kevlar est aussi appelé aramide ou polyamide aromatique synthétique.  Nous 
	pouvons définir cela comme suit:
	\begin{enumerate}
	\item Synthétique, fait en laboratoire, ne vient pas de la nature.
	\item Aromatique, possède une structure circulaire et très résistante, comme le benzène.
	\item Polyamide, les structures circulaires se joignent entre elles pour former des chaînes très longues
	\item Polymère, le matériaux est constitué de molécules identiques liées entres elles.
	\end{enumerate}
	
\subsection{}
Etant donné que le Kevlar est un polymère, il y a assemblage de molécules identiques.  Pour le Kevlar, il s'agit d'un 
groupe amide et d'un groupe phényle.

	%\begin{center}
	%	\includegraphics[scale=0.5]{Schema/formulechimique.png}  
	%\end{center}
	
	Ces monomères se lient les uns aux autres pour former des longues chaînes, mais entre les deux groupes cités 
	çi-dessus( groupe amide (carbonyle) et groupe phényle)%faire un vraie citation
	, il y a une disposition spéciale des molécules, les \ce{O} et les \ce{H} se disposent de part et 
	d'autres de la chaîne.  Cela est nécessaire pour pouvoir former des ponts H entre les différents monomères, si les 
	C et le H étaient du même côtés de la chaîne, il y aurait trop peu d'espace pour que les ponts H puissent se former 
	et dès lors la molécule perdrait beaucoup de résistance.
	

\subsection{}
La force et la résistance du Kevlar vient surtout de sa composition interne.  Les ponts hydrogènes sont formés entre 
l'oxygène dense en électron et l'hydrogène déficient en électron.  Grâce au fait que les atomes d'oxygène et 
d'hydrogène se situe de part et d'autre du groupe phényle, il y a des ponts hydrogènes formés à intervalles réguliers.  
Par la résistance de ces liaisons et par leur dispersion très homogène, on sait maintenant pourquoi le Kevlar est très 
résistant!  On peut également ajouter que les polymères de Kevlar ont une orientation radiale ( c'est à dire que la
longueur de la fibre est parallèle au tissus), cela donne une certaine régularité à la structure interne des fibres.
	


\section{Les effets de l'eau}
\subsection{Liaison hydrogène}
Le groupe carbonyle ainsi que l'hydrogène sont présents dans chaque monomère de ce fait, les liaisons hydrogènes se font
de manière récurrente et multiple à travers les longues chaînes de molécules.

Cependant ces liaisons hydrogènes sont aussi une faiblesse du Kevlar, en effet au contact de l'eau ces liaisons ont 
tendance à se casser pour privilégier des liaisons hydrogènes entre l'hydrogène du Kevlar et l'oxygène de l'eau.
\subsection{Les paramètres mécaniques}
Des recherches ont montré les effets de l'eau sur l'allongement à la rupture, la force à la rupture, le module de Young ainsi que le coefficient de poisson. Les résultats de ces recherches sont répertoriées dans le tableau~\ref{mecaniqueau}
\begin{figure}
\centering
\begin{tabular}{|c|p{2.6cm}|p{2.6cm}|p{2.6cm}|p{2.6cm}|}
\hline 
Conditions&L'allongement à la rupture(\%)&La force à la rupture(GPa)&le module de Young(GPa)&Le coefficient de poisson\\ 
\hline 
Non traité&$3.7\pm 0.2$&$3.66\pm 0.18 $&$93\pm 3$&$0.24$\\ 
\hline 
Traité dans l'eau&$3.6\pm 0.3$&$3.08\pm 0.35$&$80\pm 6$&$0.24$\\ 
\hline 
\end{tabular} 
\caption{Les propriétés mécaniques d'une seule fibre de kevlar T290}
\label{mecaniqueau}
\end{figure}
