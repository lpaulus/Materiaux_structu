\section{Propriétés physiques et chimiques du matériaux}
\subsection{Disposition générale de la molécule}%c'est pas un peu pourri comme titre ? :p

%	Ces monomères se lient les uns aux autres pour former des longues chaînes. Entre le groupe amide et le groupe phényle, les molécules se disposent d'une manière spéciale, les \ce{O} et les \ce{H} se disposent de part et d'autres de la chaîne.  Cela est nécessaire pour pouvoir former des liaisons hydrogènes entre les différents monomères, si les \ce{C} et le \ce{H} étaient du même côtés de la chaîne, il y aurait trop peu d'espace pour que les liaisons hydrogènes puissent se former et dès lors la molécule perdrait beaucoup de résistance.
%	
%	 On peut également ajouter que les polymères de Kevlar ont une orientation radiale\footnote{c'est à dire que la longueur de la fibre est parallèle au tissus} et que c'est un matériau anisotrope\footnote{La résistance dans le sens de la fibre n'est pas la même que perpendiculairement à la fibre}.  A cause de cette géométrie structurelle, on comprend que la reprise en traction soit bien plus grande que la reprise en compression.  
	


	Le Kevlar est aussi appelé aramide ou polyamide aromatique synthétique.  Nous 
	pouvons définir cela comme suit:
	\begin{enumerate}
	\item Synthétique, fait en laboratoire, ne vient pas de la nature.
	\item Aromatique, possède une structure circulaire et très résistante, comme le benzène.
	\item Polyamide, les structures circulaires se joignent entre elles pour former des chaînes très longues
	\item Polymère, le matériaux est constitué de molécules identiques liées entres elles.
	\end{enumerate}
	

Étant donné que le Kevlar est un polymère, il y a assemblage de molécules identiques.  Pour le Kevlar, il s'agit d'un 
groupe amide et d'un groupe phényle.

	%\begin{center}
	%	\includegraphics[scale=0.5]{Schema/formulechimique.png}  
	%\end{center}
	
	Ces monomères se lient les uns aux autres pour former des longues chaînes, mais entre le groupe amide et le groupe phényle
	, il y a une disposition spéciale des molécules, les \ce{O} et les \ce{H} se disposent de part et 
	d'autres de la chaîne.  C'est nécessaire pour pouvoir former des liaisons hydrogènes entre les différents monomères, si les 
	\ce{C} et le \ce{H} étaient du même côté de la chaîne, il y aurait trop peu d'espace pour que les liaisons hydrogènes puissent se former.



		
\subsection{Résistance à la traction}
	\begin{table}
	\centering
		\begin{tabular}{|c|p{5cm}|p{5cm}|p{5cm}|p{5cm}|}
		\hline

		Matériau & Résistance à la traction $[MPa]$ & Masse volumique $[\unit{}{g\per cm^3}]$& 				Résistance spécifique $[\unit{}{Mpa cm^3 \per g}]$\\

		\hline
		Kelvar 49 & 3600-4100 & 1.44 & 2700\\
		\hline
		Carbone & 5700 & 1.5 & 3800 \\
		\hline
		Acier & 760 & 7.8 & 97 \\
		\hline
		Béton & 3 & 2.5 & 1.3 \\
		\hline
		\end{tabular}
	\caption{Propriétés mécaniques de différents matériaux}
	\label{tab:mecatab}
	\end{table}

Nous pouvons observer dans le tableau~\ref{tab:mecatab} qu’avec une résistance à la traction de plus de \unit{3600}{MPa} le kevlar possède une résistance cinq fois supérieure à celle de l’acier tout en étant plus léger. En effet, la masse volumique de kevlar est de $\unit{1.44}{g\per cm^3}$. Ce qui fait que la résistance spécifique\footnote{La résistance spécifique d’un matériau est sa résistance par rapport à sa masse volumique.} du kevlar est 28.7 fois supérieur à celle de l’acier. Nous constatons donc qu'à poids égal, le Kevlar est bien plus intéressant au niveau de la reprise d'efforts en traction que l'acier. Toutefois, le carbone supplante le Kevlar en traction, même si le carbone a une masse volumique légèrement plus importante que le Kevlar sa résistance en traction est bien plus élevée.


%La résistance à la traction du Kevlar vient surtout de sa composition interne.  
%Les liaisons hydrogènes sont formées entre 
%l'oxygène dense en électron et l'hydrogène déficient en électron.  Grâce au fait que les atomes d'oxygène et 
%d'hydrogène se situent de part et d'autre du groupe phényle, il y a des liaison hydrogènes formés à intervalles réguliers.  
%Par la résistance de ces liaisons et par leur dispersion très homogène, on sait maintenant pourquoi le Kevlar est très 
%résistant!  On peut également ajouter que les polymères de Kevlar ont une orientation radiale\footnote{La
%longueur de la fibre est parallèle au tissus}, et que c'est un matériau anisotrope\footnote{La résistance dans le sens de la fibre n'est pas la même que perpendiculairement à la fibre}.  A cause de cette géométrie structurelle, on comprend que la reprise en traction soit bien plus grande que la reprise en compression.

Pour expliquer pourquoi le Kevlar a une très bonne reprise en traction, il faut s'intéresser à la structure chimique du matériau la force et la résistance du Kevlar venant surtout de sa composition interne.  Le Kevlar est formé de longues chaînes de molécules alignées parallèlement les unes par rapport aux autres procurant 
ainsi une forte résistance à la traction. Cette rigidité est due à la structure de base du kevlar. En effet, elle vient 
de la position en para du noyau benzénique qui est le squelette de la molécule.  Une autre propriété du Kevlar qui 
contribue à sa résistance en traction sont les liaisons hydrogènes\footnote{La liaison hydrogène est une attraction 
dipôle-dipôle forte. Elle se présente lorsqu'on a un (ou plusieurs) atome d'hydrogène lié à un atome fortement
électromagnétique. C'est une liaison forte.} qui se passent entre l'hydrogène chargé positivement et le groupe 
carbonyle.  Dans le cas du Kevlar, les liaisons hydrogènes sont formées entre l'oxygène dense en électron et l'hydrogène déficient en électron.  Grâce au fait que les atomes d'oxygène et d'hydrogène se situe de part et d'autre du groupe phényle, il y a des liaisons hydrogènes formées à intervalles réguliers et cela permet d'obtenir une homogénéité dans la chaîne et donc d'avoir une résistance homogène sur le matériau.
De plus, les polymères constituant le Kevlar ont une orientation radiale\footnote{La longueur de la fibre est parallèle au tissus.}
\subsection{Résistance à la compression}
Un des défauts majeur du Kevlar est sa résistance en compression. En effet, le kevlar étant un matériaux anisotrope, celle-ci représente uniquement un dixième de sa résistance à la traction. La faible tenue mécanique en compression est généralement attribuée à une mauvaise adhérence des fibres à la matrice dans le matériau composite, on lui associe même un comportement ductile et non linéaire. Lors de la compression des fibres de Kevlar, celles-ci se plient et fléchissent comme le ferait un spaghetti sur lequel on pousse à chacune des ses extrémités, c'est le phénomène de flambement. Pour effectuer ces tests, il faut prendre une matrice de fibres soumise à la compression à ses extrémités. De plus, pour des contraintes allant de \unit{0.3}{\%} à \unit{0.5}{\%}, on remarque la formation de bande de plis, ce sont des défauts qui résultent du flambement que produit la compression. Une illustration de ce phénomène se trouve à la FIGURE LEA où l'on peut distinguer des traits obliques discontinus. %(photo facebook)
%\begin{figure}[h]
%\begin{center}
%\includegraphics[scale=0.3]{Schema/Kinkbands.png} 
%\end{center}
%\end{figure} 
%Manque de chimie (spaguetti) 
%A completer je fais ca demain !
% Sources : https://polycomp.mse.iastate.edu/files/2012/01/3-Aramid-Fibers.pdf
\subsection{Résistance à la chaleur}
La résistance à la chaleur varie suivant l'épaisseur du fil pris en considération. Il se décompose à partir de \unit{450}{\celsius}. Une longue exposition à la température réduit la résistance à la traction, diminue son module d’élasticité et son élongation jusqu’à la rupture.  La température maximum recommandée pour une utilisation du kevlar dans l’air est de \unit{177}{\celsius}. Par contre, le kevlar offre une bonne résistance à l'enflammement.
%chimie
\begin{wrapfigure}{r}{0.18\textwidth}
 \vspace{-10pt}
\centering
\includegraphics[scale=0.6]{Schema/temperature.jpg} 
\vspace{-10pt}
\caption{A METTRE}
\vspace{-15pt}
\label{temperature}
\end{wrapfigure} 

\begin{wrapfigure}{l}{0.18\textwidth}
 \vspace{-20pt}
\centering
\includegraphics[scale=0.2]{Schema/temperature3.jpg} 
\vspace{-30pt}
\caption{A METTRE}
\vspace{-1pt}
\label{temperature3}
\end{wrapfigure} 



Le coefficient de dilatation thermique du kevlar est nul. Cela veut dire que quelle que soit la température, le volume du kevlar restera constant.

\begin{wrapfigure}{r}{0.18\textwidth}
 \vspace{-20pt}
\centering
\includegraphics[scale=0.3]{Schema/dilatationterm.jpg} 
\vspace{-10pt}
\caption{A METTRE}
\vspace{-20pt}
\label{dilatationerm} 
\end{wrapfigure} 
 

On voit sur la figure~\ref{dilatationerm} que le coefficient de dilatation du Kevlar sur la partie longitudinale est négatif, ce qui veut dire qu'on a beau augmenter la température, le Kevlar ne se dilatera pas dans le sens de la longueur.



\begin{table}
\centering
\begin{tabular}{|c|c|c|c|}
\hline

matériau & ténacité $[MPa]$ & module $[GPa]$ & élongation à la rupture $[\%]$ \\

\hline
Kelvar 29 & 2920 & 70 & 3.6 \\
\hline
Kelvar 49 & 3000 & 112 & 2.4 \\
\hline
Carbone & 3100 & 220 & 1.4\\
\hline
Acier & 1965 & 200 &  2\\
\hline
\end{tabular}
\caption{QUE REPRESENTE CE TABLEAU}
\label{Atab}
\end{table}


Comme nous pouvons constater sur le tableau~\ref{Atab} que le kevlar possède un module d’élasticité qui varie entre \unit{70}{GPa} et \unit{110}{GPa}. Le kevlar a également une haute ténacité\footnote{La ténacité caractérise le comportement d’un matériau à la rupture en présence d’une entaille}.



\subsection{Resistance aux UV}
L’un de ses principaux défauts est sa résistance aux UV. Nous pouvons en effet constater sur la figure~\ref{UVkevlar} qu’après plusieurs centaines d’heures d’expositions aux UV, le kevlar perd une bonne partie de sa résistance.


\begin{wrapfigure}{r}{0.18\textwidth}
 \vspace{-20pt}
\centering
\includegraphics[scale=0.3]{Schema/UVkevlar.jpg} 
\vspace{-10pt}
\caption{Effet des UV sur le kevlar}
\vspace{-20pt}
\label{UVkevlar}
\end{wrapfigure} 


\subsection{Résistance au PH}

Le Kevlar est virtuellement non affecté par un PH de 7, c’est-à-dire un PH neutre.  Cependant, sa résistance diminue dans les milieux basiques et acides.  La résistance décroit au fur et à mesure qu’on s’éloigne du PH neutre.  Mais le Kevlar résiste mieux aux solutions basiques qu’aux solutions acides.   Voilà un schéma qui montre l’évolution de la résistance du Kevlar plongé dans une solution pendant \unit{16}{h}.

%\begin{center}
%\includegraphics[scale=0.5]{phphoto.png} 
%\end{center}
\begin{figure}[h]
\centering
\includegraphics[scale=0.5]{Schema/phkevlar.jpg}  
\caption{Résistance du Kevlar au pH}
\label{phkevlar}
\end{figure}

Riewald et al. ont réalisé une étude concernant la résistance des fibres Kevlar 29 et 49 et il en découle que les fibres perdent \unit{1.5}{\%} de leur résistance mécanique après un an d’immersion dans l’eau de mer.




%\subsection{Qualités}
%Qu'en est-il des qualités du Kevlar ? Le Kevlar est utilisé dans une multitude d'applications pour ses propriétés de résistance, citons par exemple: les gilets pare-balles, dans le domaine automobile, aéronautique, ainsi que dans la structure des voiles de bateaux. Etant un matériaux cristallin, le Kelvlar est caractérisé par une rigidité et une résistance à la rupture exceptionnelle pour un polymère pour un poids relativement léger. De plus, il ne fond pas, mais se décompose juste pour des températures excédant les 400\degree à 450\degree. C'est un matériau fort et robuste qui à l'avantage d’être léger, il présente un bon module d'élasticité ainsi qu'un bon allongement sous charge. 
%
%\subsection{Défauts}
%Le Kevlar présente aussi des défauts, notamment des perturbations dues à l'humidité, c'est pourquoi il est généralement traîté en étuve. De plus, on rencontre des difficultés lorqu'il faut le couper ou encore le dimensionner. En effet, il présente une résistance à la coupure qui est de l'ordre de 5 fois celle du cuir.

%\section{Production}
%Le Kevlar est synthétisé à partir de monomères 1,4-phenyl-diamine et de terephthaloyl chloride.\\
%On le trouve a un prix élevé sur le marché à cause des difficultés dues à l'utilisation d'acide sulfurique concentré lors de sa production. 
%	


%\section{Sites}
%-http://www.ansellpro.com/auto/faq2.asp
%-http://www.dupont.com/products-and-services/fabrics-fibers-nonwovens/fibers/articles/kevlar-properties.html
%-http://www.explainthatstuff.com/kevlar.html



	

\subsection{Résistance à l'eau}
Le groupe carbonyle ainsi que l'hydrogène sont présents dans chaque monomère de ce fait, les liaisons hydrogènes se font
de manière récurrente et multiple à travers les longues chaînes de molécules.

Cependant ces liaisons hydrogènes sont aussi une faiblesse du Kevlar, au contact de l'eau ces liaisons ont 
tendance à se casser pour privilégier des liaisons hydrogènes entre l'hydrogène du Kevlar et l'oxygène de l'eau définissant ainsi le comportement de reprise d'humidité du matériau.

Des recherches ont montré les effets de l'eau sur l'allongement à la rupture, la force à la rupture, le module de Young ainsi que le coefficient de poisson. Les résultats de ces recherches sont répertoriées dans le tableau~\ref{mecaniqueau}.
\begin{table}
\centering
\begin{tabular}{|c|p{2.6cm}|p{2.6cm}|p{2.6cm}|p{2.6cm}|}
\hline 
Conditions&L'allongement à la rupture(\%)&La force à la rupture(GPa)&le module de Young(GPa)&Le coefficient de poisson\\ 
\hline 
Non traité&$3.7\pm 0.2$&$3.66\pm 0.18 $&$93\pm 3$&$0.24$\\ 
\hline 
Traité dans l'eau&$3.6\pm 0.3$&$3.08\pm 0.35$&$80\pm 6$&$0.24$\\ 
\hline 
\end{tabular} 
\caption{Les propriétés mécaniques d'une seule fibre de kevlar T290}
\label{mecaniqueau}
\end{table}

%site internet cool : http://www.christinedemerchant.com/carbon-kevlar-glass-comparison.html
\begin{figure}
\centering
\includegraphics[scale=0.65]{Schema/hydrolyse.jpg} 
\caption{Mécanisme de l’hydrolyse du PPTA}
\end{figure} 

On remarque que le module de Young est fortement diminué, montrant significativement la perte de résistance du matériaux en contact de l'eau.



%\begin{figure}
%\begin{center}
%\begin{tabular}{|c|c|p{3cm}|}
%\hline
%Température $[\unit{}{\celsius}]$ & Temps $[\unit{}{jour}]$ & Dégradation de la résistance mécanique $[\unit{}{\%\per an}]$\\
%\hline
%200 & 2 & 12426\\
%\hline
%175 & 4 & 5585  \\
%\hline
%175 & 4 & 5466 \\
%\hline
%150 & 14 & 1563  \\
%\hline
%125 & 32 & 199\\
%\hline
%125 & 40 & 264  \\
%\hline
%100 & 162 & 64 \\
%\hline
%100 & 281 & 76-86  \\
%\hline
%\end{tabular}
%\caption{Taux de dégradation hydrolytique de la 
%résistance mécanique de fibres Kevlar 49 
%en fonction de la température à 100\% d'humidité relative.}
%\end{center}
%\end{figure}


%\begin{figure}[h]
%\centering
%\includegraphics[scale=0.6]{Schema/hydrolyse2.jpg} 
%\caption{Taux de dégradation hydrolytique de la 
%résistance mécanique de fibres Kevlar 49 
%en fonction de la température à 100\% d'humidité relative.}
%\label{hydrolyse2}
%\end{figure} 
