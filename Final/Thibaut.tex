\section{Application : voiles}

%\subsection{Introduction}
%   Le kevlar est un matériau utilisé dans une grande variété de domaines.  Parmi les applications les plus connues il y a les gilets pare-balles, des parties d’ailes d’avion, la coque des bateaux,... Comme application pour notre matériau nous avons décidé de nous intéresser à l’utilisation du Kevlar pour les voiles de voiliers et plus particulièrement les voiliers de compétition.  Grâce à ses propriétés physiques et chimiques incroyables, le Kevlar est toujours très largement utilisé dans les compétitions nautiques et cela, à juste titre.

\subsection{Le Kevlar, oui mais quel Kevlar?}
%\begin{wrapfigure}{L}{0.20\textwidth}
% \vspace{-10pt}
%\centering
%\includegraphics[scale=0.3]{Schema/kevlar_taffeta2.jpg} 
%\vspace{-10pt}
%\end{wrapfigure}
Tout d’abord, pourquoi se focaliser sur les voiliers de compétition et non les voiliers de plaisance ?  Simplement parce que le Kevlar présente des caractéristiques formidables pour la performance mais il présente des gros inconvénients pour des côtés plus pratiques et financiers. Penchons-nous donc plus en détails sur les propriétés de ce matériau. Nous nous focaliserons surtout sur le Kevlar 49 et le Kevlar 29. En effet, ces deux types de Kevlar sont principalement utilisés pour les voiles tandis que d’autres, tels que le 129, ne sauraient être utilisés pour faire une voile compétitive.

Le Kevlar 49 est 50\% plus résistant que le Kevlar 29 mais moins flexible. En ce qui concerne la voile, un bon équilibre entre la résistance et la flexibilité est nécessaire. Effectivement, une voile trop flexible va se déformer plus vite et donc apporter une résistance moindre au vent, entraînant une perte de propulsion du navire.  Un autre inconvénient d'une voile trop flexible est le fait qu'elle fasseye\footnote{Faseyer : En parlant d'une voile, flotter, battre au vent. Ce phénomène apparait lorsque la voile n'est pas assez tendue ou quand le vent n'est pas assez fort ou mal orienté. Il y a donc de la surface qui flotte comme du linge qui pend} et réduit la capacité totale de propulsion.  Au contraire, une voile trop peu flexible (Kevlar 129) va avoir une surface au vent moindre (car peu de déformation) et donc une surface de voile plus petite.  Un bon équilibre entre les deux va transformer plus efficacement la force du vent en vitesse. Il y a peu, la société DuPont a créé un nouveau Kevlar appelé le Kevlar Edge.  C’est un Kevlar hybride qui allie les points forts du Kevlar 29 et 49.  Le Kevlar Edge est 25\% plus flexible que le Kevlar 49 pour une résistance égale.  Cette nouvelle variante allie les avantages des deux modèles précédents. 

\subsection{Les avantages}
Le Kevlar a beaucoup de propriétés géniales pour la voile :  c’est un matériau extrêmement résistant (5 fois le module de l’acier à poids égal) et très léger.  Dans le domaine de la voile, il y a deux données principales à prendre en compte pour la vitesse, à savoir la propulsion et la traînée.  En allégeant au maximum le poids du voilier, la traînée sera réduite. En ayant une bonne résistance et flexibilité de la voile, la propulsion sera efficace.  Ensuite, le Kevlar a une très bonne résistance en traction. Ceci est un atout majeur sachant que le vent gonfle les voiles. Le Kevlar n’offrant pas de dilatation thermique notable, cela permet d’avoir une voile apportant les mêmes propriétés par tous les temps.  De plus, le fait que le Kevlar soit moins sensible en milieu alcalin qu’en milieu acide apporte un avantage énorme en compétition sur eau de mer. Puisque la mer est légèrement alcaline, la voile ne sera pas affectée par des éclaboussures ou autre mouillage, facteur primordial lorsqu’on passe plusieurs semaines en mer. 

\subsection{Les inconvénients}
Parmi les inconvénients du Kevlar il y a sa faible résistance aux rayonnements UV.  La couleur d’origine sable devient noir et la voile s’en trouve affectée. L’exposition aux UV, au soleil donc, affaiblit la souplesse et la solidité de la voile.  Celle-ci est aussi sensible au fluage, c’est-à-dire au pliage.  Le fluage endommage les voiles et il faut donc des véhicules adaptés pour les transporter, en raison de leur encombrement; même en les roulant, les voiles peuvent garder une taille assez conséquente.  On peut aussi citer une reprise d’humidité importante (4\%) qui va alourdir la voile quand celle-ci est mouillée et diminuer sa résistance.  L’usinage du Kevlar est assez complexe et donc le coût de production va s’en trouver augmenté.
\subsection{Alternative}
Une bonne alternative du Kevlar est le carbone. En effet, celui-ci présente une meilleure résistance en traction que le Kevlar et est donc tout autant utilisé dans le monde de la voile de compétition.  Il possède également un gros avantage par rapport au Kevlar : il est virtuellement insensbible aux UV, ce qui permet d'avoir une voile plus durable face au soleil. Les coûts d'entretien sont donc réduit à ce niveau-là. De plus, le coût du carbone est beaucoup plus faible que celui du Kevlar. Cependant un inconvénient du carbone est qu'il a tendance à moins se déformer que le Kevlar, engendrant une prise au vent plus faible. 


\subsection{Voiles avec plusieurs matériaux}
Le Kevlar peut aussi être utilisé pour renforcer une voile contenant déjà du Kevlar ou non : pour les coutures ou pour simplement amener plus de résistance à un endroit voulu. 
\subsubsection{Les voiles laminées}
Pour les voiles laminées, on utilise plusieurs matériaux afin de bénéficier des avantages de chacun d’eux.
La fibre (Kevlar pour notre cas) est prise en sandwich entre deux couches de film parfois appelé taffetas. La fibre procure une grande résistance à l'élongation (déformation de la voile), tandis que les deux films assurent la protection de la fibre en améliorant la résistance à l'abrasion et aux déchirures. Dans de récentes évolutions, des fils d'aramide sont inclus dans le film. On obtient donc un enchaînement film-fibre-film dont les fibres reprennent la plupart des efforts de traction alors que les films protègent essentiellement les fibres.  Cette technique offre une grande polyvalence pour la fabrication de voile.  On peut avoir une couche de tissu protectrice ou bien deux (pour alléger la voile ou le coût). 
\subsubsection{Fabrication en 3DL} Pour fabriquer ce type de voile, la technique utilisée est appelée 3DL.  Tout d’abord la voile est modélisée sur un ordinateur en 3D, ensuite, un programme lit le plan pour enfin envoyer les instructions à un robot qui fabrique sur mesure la base de la voile.  Des machines cousent les fils sur la base, donnant la première couche.  Il est primordial de mettre la même tension dans chaque fil pour que la voile ait une résistance uniforme.   L'étape suivante est un laminage (collage les fils et le film) pour bien fixer les deux couches.   Ensuite,la dernière couche est recouverte par un film qui sera par la suite tendu et fixé grâce à un grand sac sous vide qui va appliquer une très grande force pour sceller les différentes parties de la voile.  Pour finir, la voile est chauffée à une température bien spécifique pour fixer une dernière fois les différentes couches entre elles.
\subsubsection{Tissage}
Il est également possible de tisser une voile avec du fil de kevlar et de carbone.  Près des anneaux, il faut une très grande résistance pour que la voile  ne se déchire pas.  Le kevlar intervient donc et il y aura plus de tissage avec une grande quantité de kevlar dans cette partie-là.


