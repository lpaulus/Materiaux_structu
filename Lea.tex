%Léa's part
\chapter{Structure chimique}

Le Kevlar est formé de longues chaînes de molécules alignées parallèlement les unes par rapport aux autres procurant 
ainsi une forte résistance à la traction. Cette rigidité est due à la structure de base du kevlar, en effet elle vient 
de la position en para du noyau benzénique qui est le squelette de la molécule.  Une autre propriété du Kevlar qui 
contribue à sa résistance en traction sont les liaisons hydrogènes\footnote{La liaison hydrogène est une attraction 
dipôle-dipôle forte. Elle se présente lorsqu'on a un (ou plusieurs) atome d'hydrogène lié à un atome fortement
électromagnétique. C'est une liaison forte.} qui se passent entre l'hydrogène chargé positivement et le groupe 
carbonyle.
\section{Les effets de l'eau}
\subsection{Liaison hydrogène}
Le groupe carbonyle ainsi que l'hydrogène sont présents dans chaque monomère de ce fait, les liaisons hydrogènes se font
de manière récurrente et multiple à travers les longues chaînes de molécules.

Cependant ces liaisons hydrogènes sont aussi une faiblesse du Kevlar, en effet au contact de l'eau ces liaisons ont 
tendance à se casser pour privilégier des liaisons hydrogènes entre l'hydrogène du Kevlar et l'oxygène de l'eau.
\subsection{Les paramètres mécaniques}
Des recherches ont montré les effets de l'eau sur l'allongement à la rupture, la force à la rupture, le module de Young ainsi que le coefficient de poisson. Les résultats de ces recherches sont répertoriées dans le tableau~\ref{mecaniqueau}
\begin{figure}
\centering
\begin{tabular}{|c|p{2.6cm}|p{2.6cm}|p{2.6cm}|p{2.6cm}|}
\hline 
Conditions&L'allongement à la rupture(\%)&La force à la rupture(GPa)&le module de Young(GPa)&Le coefficient de poisson\\ 
\hline 
Non traité&$3.7\pm 0.2$&$3.66\pm 0.18 $&$93\pm 3$&$0.24$\\ 
\hline 
Traité dans l'eau&$3.6\pm 0.3$&$3.08\pm 0.35$&$80\pm 6$&$0.24$\\ 
\hline 
\end{tabular} 
\caption{Les propriétés mécaniques d'une seule fibre de kevlar T290}
\label{mecaniqueau}
\end{figure}

\section{Le PH}
Au plus le PH est acide au plus les conditions sont favorables à la protonisation.
Un PH acide fournit des conditions favorables pour la protonation de la fonction amine du polymère.

\subsection{}
	Le Kevlar peut être considéré comme un plastique très résistant.  On peut voir le Kevlar comme une très grande 
	chaîne de différents plastiques.
	Le Kevlar est aussi appelé aramide ou polyamide aromatique synthétique.  Nous 
	pouvons définir cela comme suit:
	\begin{enumerate}
	\item Synthétique, fait en laboratoire, ne vient pas de la nature.
	\item Aromatique, possède une structure circulaire et très résistante, comme le benzène.
	\item Polyamide, les structures circulaires se joignent entre elles pour former des chaînes très longues
	\item Polymère, le matériaux est constitué de molécules identiques liées entres elles.
	\end{enumerate}
	
\subsection{}
Etant donné que le Kevlar est un polymère, il y a assemblage de molécules identiques.  Pour le Kevlar, il s'agit d'un 
groupe amide et d'un groupe phényle.

	%\begin{center}
	%	\includegraphics[scale=0.5]{Schema/formulechimique.png}  
	%\end{center}
	
	Ces monomères se lient les uns aux autres pour former des longues chaînes, mais entre les deux groupes cités 
	çi-dessus( amide et phenyl)%faire un vraie citation
	, il y a une disposition spéciale des molécules, les \ce{O} et les \ce{H} se disposent de part et 
	d'autres de la chaîne.  Cela est nécessaire pour pouvoir former des ponts H entre les différents monomères, si les 
	C et le H étaient du même côtés de la chaîne, il y aurait trop peu d'espace pour que les ponts H puissent se former 
	et dès lors la molécule perdrait beaucoup de résistance.
	

\subsection{}
La force et la résistance du Kevlar vient surtout de sa composition interne.  Les ponts hydrogènes sont formés entre 
l'oxygène dense en électron et l'hydrogène déficient en électron.  Grâce au fait que les atomes d'oxygène et 
d'hydrogène se situe de part et d'autre du groupe phényle, il y a des ponts hydrogènes formés à intervalles réguliers.  
Par la résistance de ces liaisons et par leur dispersion très homogène, on sait maintenant pourquoi le Kevlar est très 
résistant!  On peut également ajouter que les polymères de Kevlar ont une orientation radiale ( c'est à dire que la
longueur de la fibre est parallèle au tissus), cela donne une certaine régularité à la structure interne des fibres.
	


