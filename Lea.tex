%Léa's part
\chapter{Structure chimique}

Le Kevlar est formé de longues chaines de molécules alignées parallèlement les unes par rapport aux autres procurant 
ainsi une forte résistance à la traction. Cette rigidité est due à la structure de base du kevlar, en effet elle vient 
de la position en para du noyau benzénique qui est le squelette de la molécule.  Une autre propriété du Kevlar qui 
contribue à sa résistance en traction sont les liaisions hydrogènes\footnote{La liaison hydrogène est une attraction 
dipôle-dipôle forte. Elle se présente lorsqu'on a un (ou plusieurs) atome d'hydrogène lié à un atome fortement
électromagnétique. C'est une liaison forte.} qui se passent entre l'hydrogène chargé positivement et le groupe 
carbonyle.

\section{Liaison hydrogène}
Le groupe carbonyle ainsi que l'hydrogène sont présent dans chaque monomère de ce fait, les liaisons hydrogènes se font
de mainière récurente et multiple à travers les longues chaines de molécules.

Cependant ces liaisions hydrogènes sont aussi une faiblesse du Kevlar, en effet au conctact de l'eau ces liasions ont 
tendance à se casser pour priviligier des lisasions hydrogènes entre l'hydrogène du Kelvlar et l'oxygène de l'eau.


\subsection{}
	Le Kevlar peut être considéré comme un plasique très résistant.  On peut le voir le Kevlar comme une très grande 
	chaîne de différents de plastique.  Le Kevlar est aussi appelé aramide ou polyamide aromatique synthetique.  Nous 
	pouvons définir cela comme suit:
	\begin{enumerate}
	\item Synthetique: Fait en laboratoire, ne vient pas de la nature
	\item Aromatique: Possède une structure circulaire et très résistante, comme le benzene
	\item Polyamide: Les structure circulaire se joignent entre elles pour former des chaines très longues
	\item Polymere: Le matériaux est constitué de molécules identiques liées entres elles.
	\end{enumerate}
	
\subsection{}
Etant donné que le Kevlar est un polymere, il y a assemblage de molécules identiques.  Pour le Kevlar, il s'agit d'un 
groupe amide et d'un groupe phenyle.

	%\begin{center}
	%	\includegraphics[scale=0.5]{Schema/formulechimique.png}  
	%\end{center}
	
	Ces monomeres se lient les uns aux autres pour former des longues chaines, mais entre les deux groupes cités 
	çi-dessus( amide et phenyl), il y a une disposition spéciale des molécules, les O et les H se disposent de part et 
	d'autres de la chaine.  Cela est necessaire pour pouvoir former des ponts H entre les differents monomères, si les 
	C et le H étaient du même côtés de la chaine, il y aurait trop peu d'espace pour que les ponts H puissent se former 
	et dès lors la molecule perdrait beaucoup de resistance.
	

\subsection{}
La force et la résistance du Kevlar vient surtout de sa composition interne.  Les ponts hydrogènes sont formés entre 
l'oxygène dense en électron et l'hydrogène déficient en électron.  Grâce au fait que les atomes d'oxygène et 
d'hydrogène se situe de part et d'autre du groupe phenyle, il y a des ponts hydrogènes formés à intervals réguliers.  
Par la résistance de ces liaisons et par leur dispersion très homogène, on sait maintenant pourquoi le Kevlar est très 
résistant!  On peut également ajouter que les polymères de Kevlar ont une orientation radiale ( c'est à dire que la
longueur de la fibre est parallèle au tissus), cela donne une certaine regularité à la structure interne des fibres.
	


