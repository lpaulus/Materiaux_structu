%Léa's part
\chapter{Structure chimique}

Le Kevlar est formé de longues chaines de molécules alignées parallèlement les unes aux autres procurant ainsi une forte résistance à la traction. 
Cette rigidité est due à la structure de base du kevlar, en effet elle vient de la position en para du noyau benzénique qui est le squelette de la molécule.
Une autre propriété du Kevlar qui contribue à sa résistance en traction sont les ponts hydrogènes qui se passent entre l'hydrogène chargé positivement et le groupe carbonyle.
\section{Pont hydrogène}
Le groupe carbonyle ainsi que l'hydrogène sont présent dans chaque monomère de ce fait, les ponts hydrogènes se font de mainière récurente et multiple à travers les longue chaines de molécules. 

