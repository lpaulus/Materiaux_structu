%Léa's part
\chapter{Structure chimique}

Le Kevlar est formé de longues chaines de molécules alignées parallèlement les unes aux par rapport autres procurant ainsi une forte résistance à la traction. 
Cette rigidité est due à la structure de base du kevlar, en effet elle vient de la position en para du noyau benzénique qui est le squelette de la molécule.
Une autre propriété du Kevlar qui contribue à sa résistance en traction sont les liaisions hydrogènes\footnote{La liaison hydrogène est une attraction dipôle-dipôle forte. Elle se présente lorsqu'on a un (ou plusieurs) atome d'hydrogène lié à un atome fortement électromagnétique. C'est une liaison forte.} qui se passent entre l'hydrogène chargé positivement et le groupe carbonyle.
\section{Liaison hydrogène}
Le groupe carbonyle ainsi que l'hydrogène sont présent dans chaque monomère de ce fait, les liaisons hydrogènes se font de mainière récurente et multiple à travers les longues chaines de molécules.

Cependant ces liaisions hydrogènes sont aussi une faiblesse du Kevlar, en effet en conctact de l'eau ces liasions ont tendance à se casser pour priviligier des lisasion hydrogènes entre l'hydrogène du Kelvlar et l'oxygène de l'eau.


