%coco's part

\documentclass[10pt,a4paper]{article}
\usepackage[utf8]{inputenc}
\usepackage[francais]{babel}
\usepackage[T1]{fontenc}
\usepackage{amsmath}
\usepackage{amsfonts}
\usepackage{amssymb}
\usepackage{graphicx}
\usepackage[squaren,Gray]{SIunits}
\author{Joachim Corentin}
\begin{document}
\part{KEVLAR}
\section{Propriété physique:}
Le Kevlar est une fibre synthétique avec de très bonnes propriétés mécanique en traction, à savoir, une résistance à la rupture de 3 100 MPa avec un module entre 70 et 125 GPa, ainsi qu'en fatigue. Pour donner un ordre de grandeur, en traction, le Kevlar est plus résistant que l'acier mais moins que la fibre de carbonne. Il est constitué d'une longue chaine de polymères orientés de manière parallèle. Il en existe de divers grade comme par exemple, le Kevlar 29, Kevlar 49, Kevlar 129... 

\subsection{Qualités:}
Qu'en est-il des qualités du Kevlar ? Le Kevlar est utilisé dans une multitude d'applications pour ses propriétés de résistance, citons par exemple: les gilets pare-balles, dans le domaine automobile, aéronautique, ansi que dans la structure des voiles de bateaux. Etant un matériaux cristallin, le Kelvar est caractérisé par une rigidité et une résistance à a rupture exceptionnelle pour un polymère. De plus, il ne fond pas, mais se décompose juste pour des températures exédant les 400\degree à 450\degree. C'est un matériau fort et robuste qui à l'avantage d'ètre léger et qui présente un bon module d'élasticité ainsi qu'un bon allongement sous charge.

\subsection{Défauts:}
Le Kelvar présente aussi ses défauts, notamment des perturbations dues à l'humidité, c'est pourquoi il est généralement traîté en étuve. De plus, il est diffcile à couper et à dimensionner.

\section{Production}
Le Kevlar est synthétisé à partir de monomères 1,4-phenyl-diamine et de terephthaloyl chloride.\\

	Le Kevlar a un prix élevé à cause des difficultés dues à l'utilisation d'acide sulfurique concentré durant sa production. Sulphuric acid is needed to keep the water-insoluble polymer in solution during its synthesis and spinning. 
	
\section{Heat resistance}

8. What is the level of heat resistance protection that Kevlar provides? How about Kevlar cut resistance versus cotton?

    Answer: Knitted Kevlar can vary in heat resistance based on the yarn thickness. Heavyweight Kevlar® styles can withstand 200 F for short intervals but the level of heat resistance can go as high as 400 F). Heat transfer through Kevlar is slower than that of cotton or leather, however the dissipating factor is also slower. This slow dissipation rate is critical if the end-user is handling hot objects (over 200 F) for long periods of time without any breaks between handling parts.

    Cotton is still used in many mid- range temperature (300 F to 400 F) applications because it dissipates heat much better and is a natural fiber that actually "breathes" better than man-made fibers like Kevlar.

9. How much more cut resistances does Kevlar offer over leather?

    Answer: CPPT data shows that Kevlar is up to five times more cut resistant than leather.

10. Which product offers better heat resistance, Kevlar or cotton?

    Answer: Cotton actually offers as good or better heat resistance, but Kevlar offers better flame resistance.

\section{Sites:}
-http://www.ansellpro.com/auto/faq2.asp


\end{document}