%coco's part
\chapter{Propriétés physiques}

\section{Propriété physique} 
Le Kevlar est une fibre synthétique avec de très bonnes propriétés mécanique en traction, à savoir, une résistance à la rupture de 3 100 MPa avec un module entre 70 et 125 GPa, ainsi qu'en fatigue. Pour donner un ordre de grandeur, en traction, le Kevlar est plus résistant que l'acier mais moins que la fibre de carbonne. Cependant il se comporte relativement mal en compression, ce qui ne pose aucun soucis dans notre application qui est la voile. Il est constitué d'une longue chaine de polymères orientés de manière parallèle. Il en existe de divers grade, citons par exemple le Kevlar 29, Kevlar 49, Kevlar 129... 

\subsection{Qualités}
Qu'en est-il des qualités du Kevlar ? Le Kevlar est utilisé dans une multitude d'applications pour ses propriétés de résistance, citons par exemple: les gilets pare-balles, dans le domaine automobile, aéronautique, ansi que dans la structure des voiles de bateaux. Etant un matériaux cristallin, le Kelvlar est caractérisé par une rigidité et une résistance à la rupture exceptionnelle pour un polymère pour un poids relativement léger. De plus, il ne fond pas, mais se décompose juste pour des températures exédant les 400\degree à 450\degree. C'est un matériau fort et robuste qui à l'avantage d'ètre léger, il présente un bon module d'élasticité ainsi qu'un bon allongement sous charge. 

\subsection{Défauts}
Le Kevlar présente aussi des défauts, notamment des perturbations dues à l'humidité, c'est pourquoi il est généralement traîté en étuve. De plus, on rencontre des difficultés lorqu'il faut le couper ou encore le dimensionner. En effet, il présente une résistance à la coupure qui est de l'ordre de 5 fois celle du cuir.

\section{Production}
Le Kevlar est synthétisé à partir de monomères 1,4-phenyl-diamine et de terephthaloyl chloride.\\
On le trouve a un prix élevé sur le marché à cause des difficultés dues à l'utilisation d'acide sulfurique concentré lors de sa production. 
	
\section{Heat resistance}
Qu'en est-il de la résistance à la chaleur de ce matériau ? 
La résistance à la chaleur varie suivant l'épaisseur du fil pris en concidération. Pour du kevlar de grosse épaisseur, il peut résister jusqu'à des températures atteignant les 200\degree. Le transfert de chaleur au sein du kevlar se déroule plus lentement que losrqu'on parle de coton ou encore du cuir. Cela peut s'avérer interressant dans des utilisations qui nécessite par exemple de transporter des objets ou autre à haute température.

Cependant, le coton est toujours utilisé pour une gamme moyenne de température car il dissipe mieux la chaleur et est également une fibre naturelle qui "respire" mieux que le kevlar. En matière de résistance à la chaleur, on peut dire que le kevlar et le coton se valent, cependant le kevlar offre en plus une meilleure résistance à l'enflammement.

\section{Sites}
-http://www.ansellpro.com/auto/faq2.asp
-http://www.dupont.com/products-and-services/fabrics-fibers-nonwovens/fibers/articles/kevlar-properties.html
-http://www.explainthatstuff.com/kevlar.html
