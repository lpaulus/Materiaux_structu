\documentclass{report}
\usepackage{multicol}
\usepackage[utf8]{inputenc}
\usepackage[T1]{fontenc}      
\usepackage[francais]{babel}
\usepackage{graphicx}
\usepackage{circuitikz}
\usepackage[squaren, Gray]{SIunits}
\usepackage{sistyle}
\usepackage[autolanguage]{numprint}
\usepackage{pgfplots}
\usepackage{geometry}
\usepackage{hyperref}
\usepackage{caption}
\usepackage{amsmath,amssymb,array}
\usepackage{url}
\usepackage{fancyhdr}
\usepackage{layout}
\usepackage[version=3]{mhchem}
\usepackage{array} 
\usepackage{tikz}
	\usetikzlibrary{arrows,shapes,positioning}


\newcommand{\reporttitle}{Le Kevlar}     % Titre
\newcommand{\reportauthor}{Thibaut \bsc{Cabo} \\ Adrien \bsc{Hoedenaeken} \\ Geoffroy \bsc{Jacquet} \\ Corentin \bsc{Joachim}\\ Léa \bsc{Paulus}} % Auteur
\newcommand{\reportsubject}{Rapport de projet de matériaux structuraux} % Sujet
\newcommand{\HRule}{\rule{\linewidth}{0.5mm}}
\newcommand{\copyrigh}{{\tiny \textregistered}}
\setlength{\parskip}{1ex} % Espace entre les paragraphes

\hypersetup{
    pdftitle={\reporttitle},%
     pdfauthor={\reportauthor},%
    pdfsubject={\reportsubject},%
    pdfkeywords={rapport} {vos} {mots} {clés}
}


\setlength{\headheight}{12pt}
\setlength{\headsep}{12pt}

\pagestyle{fancy}
\lhead{\leftmark{}}
\rhead{LAUCE1031 - 2015 - gr11}
\cfoot{\thepage{}}
\begin{document}
\begin{titlepage}

\begin{center}

% Upper part of the page

\textsc{\Large Université Catholique de Louvain}\\[0.5cm]

\textsc{\LARGE Rapport de projet de matériaux structuraux}\\[0.2cm]
\textsc{\LARGE LAUCE1031}\\[0.2cm]

% Title
\HRule \\[0.2cm]
{\huge \bfseries Le Kevlar}\\
\HRule \\[0.2cm]

% Author and supervisor
\begin{center}
\includegraphics[trim=0cm 2cm 0cm 2cm, clip, width=15cm]{Schema/kevlar_taffeta2.jpg}
\end{center}
\HRule \\[0 cm]
%inventer un petit texte cool :)
Dans le cadre du cours de matériaux structuraux, il nous a été demandé de faire des recherches sur un matériau présentant des caractéristiques intérressantes en terme de structure. Dans cette optique, nous nous sommes diriger vers le kevlar lorqu'il est utilisé dans la structure des voiles de bateaux. Nous allons approfondir ce sujet afin de voir les diverses propriétés qui lui sont associées  et en quoi celles-ci sont intéressantes dans l'application des voiles.

\HRule \\[0.2cm]

\begin{minipage}{0.4\textwidth}
\begin{flushleft} \large

\begin{tabular}{l l}

\emph{Auteurs:} & \\
Thibaut & \bsc{Cabo}\\
Adrien & \bsc{Hoedenaeken}\\ 
Geoffroy & \bsc{Jacquet}\\ 
Corentin & \bsc{Joachim}\\ 
Léa & \bsc{Paulus}

\end{tabular}
\end{flushleft}
\end{minipage}
\begin{minipage}{0.4\textwidth}
\begin{flushright} \large
\emph{Cours:} \\
LAUCE1031\\
\emph{Groupe:} \\
11\\
\emph{Tuteur:} \\
Sébastien \textsc{Goessens}
\end{flushright}
\end{minipage}
\vspace{0.3cm}
% Bottom of the page

\begin{minipage}{0.3\textwidth}
\begin{flushleft}
\includegraphics[height=2cm]{Schema/logo_UCL_NEW_janv2013.JPG}
\end{flushleft}
\end{minipage}
\begin{minipage}{0.3\textwidth}
\begin{center}
{\large FSA12BA}\\
{\large \today}
\end{center}
\end{minipage}
\begin{minipage}{0.3\textwidth}
\begin{flushright}
\includegraphics[height=1cm]{Schema/epl-logo.jpg}
\end{flushright}
\end{minipage}
\end{center}
\end{titlepage}
\tableofcontents


%don"t write in this one ! :)
\section{Application : voiles}

%\subsection{Introduction}
%   Le kevlar est un matériau utilisé dans une grande variété de domaines.  Parmi les applications les plus connues il y a les gilets pare-balles, des parties d’ailes d’avion, la coque des bateaux,... Comme application pour notre matériau nous avons décidé de nous intéresser à l’utilisation du Kevlar pour les voiles de voiliers et plus particulièrement les voiliers de compétition.  Grâce à ses propriétés physiques et chimiques incroyables, le Kevlar est toujours très largement utilisé dans les compétitions nautiques et cela, à juste titre.

\subsection{Le Kevlar, oui mais quel Kevlar?}
%\begin{wrapfigure}{L}{0.20\textwidth}
% \vspace{-10pt}
%\centering
%\includegraphics[scale=0.3]{Schema/kevlar_taffeta2.jpg} 
%\vspace{-10pt}
%\end{wrapfigure}
Tout d’abord, pourquoi se focaliser sur les voiliers de compétition et non les voiliers de plaisance ?  Simplement parce que le Kevlar présente des caractéristiques formidables pour la performance mais il présente des gros inconvénients pour des côtés plus pratiques et financiers. Penchons-nous donc plus en détails sur les propriétés de ce matériau. Nous nous focaliserons surtout sur le Kevlar 49 et le Kevlar 29. En effet, ces deux types de Kevlar sont principalement utilisés pour les voiles tandis que d’autres, tels que le 129, ne sauraient être utilisés pour faire une voile compétitive.

Le Kevlar 49 est 50\% plus résistant que le Kevlar 29 mais moins flexible. En ce qui concerne la voile, un bon équilibre entre la résistance et la flexibilité est nécessaire. Effectivement, une voile trop flexible va se déformer plus vite et donc apporter une résistance moindre au vent, entraînant une perte de propulsion du navire.  Un autre inconvénient d'une voile trop flexible est le fait qu'elle fasseye\footnote{Faseyer : En parlant d'une voile, flotter, battre au vent. Ce phénomène apparait lorsque la voile n'est pas assez tendue ou quand le vent n'est pas assez fort ou mal orienté. Il y a donc de la surface qui flotte comme du linge qui pend} et réduit la capacité totale de propulsion.  Au contraire, une voile trop peu flexible (Kevlar 129) va avoir une surface au vent moindre (car peu de déformation) et donc une surface de voile plus petite.  Un bon équilibre entre les deux va transformer plus efficacement la force du vent en vitesse. Il y a peu, la société DuPont a créé un nouveau Kevlar appelé le Kevlar Edge.  C’est un Kevlar hybride qui allie les points forts du Kevlar 29 et 49.  Le Kevlar Edge est 25\% plus flexible que le Kevlar 49 pour une résistance égale.  Cette nouvelle variante allie les avantages des deux modèles précédents. 

\subsection{Les avantages}
Le Kevlar a beaucoup de propriétés géniales pour la voile :  c’est un matériau extrêmement résistant (5 fois le module de l’acier à poids égal) et très léger.  Dans le domaine de la voile, il y a deux données principales à prendre en compte pour la vitesse, à savoir la propulsion et la traînée.  En allégeant au maximum le poids du voilier, la traînée sera réduite. En ayant une bonne résistance et flexibilité de la voile, la propulsion sera efficace.  Ensuite, le Kevlar a une très bonne résistance en traction. Ceci est un atout majeur sachant que le vent gonfle les voiles. Le Kevlar n’offrant pas de dilatation thermique notable, cela permet d’avoir une voile apportant les mêmes propriétés par tous les temps.  De plus, le fait que le Kevlar soit moins sensible en milieu alcalin qu’en milieu acide apporte un avantage énorme en compétition sur eau de mer. Puisque la mer est légèrement alcaline, la voile ne sera pas affectée par des éclaboussures ou autre mouillage, facteur primordial lorsqu’on passe plusieurs semaines en mer. 

\subsection{Les inconvénients}
Parmi les inconvénients du Kevlar il y a sa faible résistance aux rayonnements UV.  La couleur d’origine sable devient noir et la voile s’en trouve affectée. L’exposition aux UV, au soleil donc, affaiblit la souplesse et la solidité de la voile.  Celle-ci est aussi sensible au fluage, c’est-à-dire au pliage.  Le fluage endommage les voiles et il faut donc des véhicules adaptés pour les transporter, en raison de leur encombrement; même en les roulant, les voiles peuvent garder une taille assez conséquente.  On peut aussi citer une reprise d’humidité importante (4\%) qui va alourdir la voile quand celle-ci est mouillée et diminuer sa résistance.  L’usinage du Kevlar est assez complexe et donc le coût de production va s’en trouver augmenté.
\subsection{Alternative}
Une bonne alternative du Kevlar est le carbone. En effet, celui-ci présente une meilleure résistance en traction que le Kevlar et est donc tout autant utilisé dans le monde de la voile de compétition.  Il possède également un gros avantage par rapport au Kevlar : il est virtuellement insensbible aux UV, ce qui permet d'avoir une voile plus durable face au soleil. Les coûts d'entretien sont donc réduit à ce niveau-là. De plus, le coût du carbone est beaucoup plus faible que celui du Kevlar. Cependant un inconvénient du carbone est qu'il a tendance à moins se déformer que le Kevlar, engendrant une prise au vent plus faible. 


\subsection{Voiles avec plusieurs matériaux}
Le Kevlar peut aussi être utilisé pour renforcer une voile contenant déjà du Kevlar ou non : pour les coutures ou pour simplement amener plus de résistance à un endroit voulu. 
\subsubsection{Les voiles laminées}
Pour les voiles laminées, on utilise plusieurs matériaux afin de bénéficier des avantages de chacun d’eux.
La fibre (Kevlar pour notre cas) est prise en sandwich entre deux couches de film parfois appelé taffetas. La fibre procure une grande résistance à l'élongation (déformation de la voile), tandis que les deux films assurent la protection de la fibre en améliorant la résistance à l'abrasion et aux déchirures. Dans de récentes évolutions, des fils d'aramide sont inclus dans le film. On obtient donc un enchaînement film-fibre-film dont les fibres reprennent la plupart des efforts de traction alors que les films protègent essentiellement les fibres.  Cette technique offre une grande polyvalence pour la fabrication de voile.  On peut avoir une couche de tissu protectrice ou bien deux (pour alléger la voile ou le coût). 
\subsubsection{Fabrication en 3DL} Pour fabriquer ce type de voile, la technique utilisée est appelée 3DL.  Tout d’abord la voile est modélisée sur un ordinateur en 3D, ensuite, un programme lit le plan pour enfin envoyer les instructions à un robot qui fabrique sur mesure la base de la voile.  Des machines cousent les fils sur la base, donnant la première couche.  Il est primordial de mettre la même tension dans chaque fil pour que la voile ait une résistance uniforme.   L'étape suivante est un laminage (collage les fils et le film) pour bien fixer les deux couches.   Ensuite,la dernière couche est recouverte par un film qui sera par la suite tendu et fixé grâce à un grand sac sous vide qui va appliquer une très grande force pour sceller les différentes parties de la voile.  Pour finir, la voile est chauffée à une température bien spécifique pour fixer une dernière fois les différentes couches entre elles.
\subsubsection{Tissage}
Il est également possible de tisser une voile avec du fil de kevlar et de carbone.  Près des anneaux, il faut une très grande résistance pour que la voile  ne se déchire pas.  Le kevlar intervient donc et il y aura plus de tissage avec une grande quantité de kevlar dans cette partie-là.



%Adrien's part
\section{Propriété physique}

\paragraph{}
Le kevlar est un aramide. Le mot aramide vient de la contraction de l'anglais aromatic polyamide. Le kevlar possède des propriétés physiques très impressionnantes. Ce sont ces propriétés qui font que c’est un matériau utilisé dans diverses domaines. Il est apprécié pour sa résistance mécanique à la traction. Nous allons reprendre ci-dessous les principales caractéristiques du Kevlar comme sa résistance à la traction, son module d’élasticité,…


\begin{center}
\begin{tabular}{|c|c|c|c|c|}
\hline

matériau & résistance à ma traction $MPa$ & masse volumique & résistance spécifique & ductilité \\

\hline
Kelvar 49 & 3600-4100 & 1.44 & 2700 & 3.8\\
\hline
Carbone & 5700 & 1.5 & 3800 & 2 \\
\hline
Acier & 760 & 7.8 & 97 & \\
\hline
Béton & 3 & 2.5 & 1.3 &\\
\hline
\end{tabular}
\end{center}

\paragraph{}
Nous pouvons observer dans ce tableau qu’avec une résistance à la traction de plus de 3600 MPa le kevlar possède une résistance cinq fois supérieure à celle de l’acier en étant plus léger. En effet, la masse volumique de kevlar est d’1.44 g/cm3. Ce qui fait la résistance spécifique du kevlar est 28.7 fois supérieur à celui de l’acier. La résistance spécifique d’un matériau est sa résistance par rapport à sa masse volumique. Par contre, le kevlar possède une résistance spécifique inférieure à celle du carbone. Nous voyons aussi que le kevlar n’est pas un matériau très ductile. 

\paragraph{}
 Le kevlar est un matériau anisotrope. Cela veut dire que sa résistance dans le sens de la fibre n’est pas la même que perpendiculairement à la fibre. C’est à cause de cette anisotropie que le kevlar ou d’autres fibres ne sont généralement pas utilisé comme matériau seul mais comme renfort.
 
\paragraph{}
Le coefficient de dilatation thermique du kevlar est nul. Cela veut dire que quelle que soit la température, le volume du kevlar restera constant.  En plus de cela il est insensible aux produits chimiques mécaniques comme les graisses, huiles, solvants et au pétrole.

\begin{center}
\begin{tabular}{|c|c|c|c|}
\hline

matériau & ténacité $MPa$ & module $GPa$ & élongation avant rupture \\

\hline
Kelvar 29 & 2920 & 70 & 3.6 \\
\hline
Kelvar 49 & 3000 & 112 & 2 \\
\hline
Carbone & 3100 & 220 & \\
\hline
Acier & 1965 & 200 &  \\
\hline
\end{tabular}
\end{center}

\paragraph{}
Comme pouvons le constater le kevlar possède un module d’élasticité de minimum 70GPa. Même si cette fois il est inférieur au module de l’acier c’est quand même une bonne valeur.  Le kevlar à également une haute ténacité. La ténacité caractérise le comportement d’un matériau à la rupture en présence d’une entaille.

\paragraph{}
Par contre le kevlar a une très mauvaise résistance à la compression. La résistance à la compression du kevlar est égale à un dixième de sa résistance à la traction. La faible tenue mécanique en compression est généralement attribuée à une mauvaise adhérence des fibres à la matrice dans le matériau composite

\paragraph{}
L’un de ses principaux défauts est sa résistance aux UV. Nous pouvons en effet constater via le graphe ci-dessous qu’après plusieurs centaines d’heures d’expositions aux UV, le kevlar perd une bonne partie de sa résistance.


\begin{figure}[h]
\begin{center}
\includegraphics[scale=0.3]{Schema/UVkevlar.png} 
\end{center}
\end{figure} 

\paragraph{}
Un autre défaut du kevlar est son comportement face à la chaleur. Il se décompose à partir de 450 \degre C. Une longue exposition à la température réduit la résistance à la traction, diminue son module d’élasticité et son élongation jusqu’à la rupture. La température maximum recommandée pour une utilisation du kevlar dans l’air est de 177 \degre C.



\begin{figure}[h]
\begin{center}
\includegraphics[scale=0.3]{Schema/UVkevlar.png} 
\end{center}
\end{figure} 

\paragraph{}
Un autre défaut du kevlar est sa grande reprise d'humidité. Le kevlar absorbe rapidement l'humidité présente dans l'air. Cela est évidemment un problème car plus le kevlar absorbe l'humidité plus sa résistance diminue.

%coco's part
\chapter{Propriétés physiques}

\section{Propriété physique} 
Le Kevlar est une fibre synthétique avec de très bonnes propriétés mécanique en traction, à savoir, une résistance à la rupture de 3 100 MPa avec un module entre 70 et 125 GPa, ainsi qu'en fatigue. Pour donner un ordre de grandeur, en traction, le Kevlar est plus résistant que l'acier mais moins que la fibre de carbonne. Cependant il se comporte relativement mal en compression, ce qui ne pose aucun soucis dans notre application qui est la voile. Il est constitué d'une longue chaine de polymères orientés de manière parallèle. Il en existe de divers grade, citons par exemple le Kevlar 29, Kevlar 49, Kevlar 129... 

\subsection{Qualités}
Qu'en est-il des qualités du Kevlar ? Le Kevlar est utilisé dans une multitude d'applications pour ses propriétés de résistance, citons par exemple: les gilets pare-balles, dans le domaine automobile, aéronautique, ansi que dans la structure des voiles de bateaux. Etant un matériaux cristallin, le Kelvlar est caractérisé par une rigidité et une résistance à la rupture exceptionnelle pour un polymère pour un poids relativement léger. De plus, il ne fond pas, mais se décompose juste pour des températures exédant les 400\degree à 450\degree. C'est un matériau fort et robuste qui à l'avantage d'ètre léger, il présente un bon module d'élasticité ainsi qu'un bon allongement sous charge. 

\subsection{Défauts}
Le Kevlar présente aussi des défauts, notamment des perturbations dues à l'humidité, c'est pourquoi il est généralement traîté en étuve. De plus, on rencontre des difficultés lorqu'il faut le couper ou encore le dimensionner. En effet, il présente une résistance à la coupure qui est de l'ordre de 5 fois celle du cuir.

\section{Production}
Le Kevlar est synthétisé à partir de monomères 1,4-phenyl-diamine et de terephthaloyl chloride.\\
On le trouve a un prix élevé sur le marché à cause des difficultés dues à l'utilisation d'acide sulfurique concentré lors de sa production. 
	
\section{Heat resistance}
Qu'en est-il de la résistance à la chaleur de ce matériau ? 
La résistance à la chaleur varie suivant l'épaisseur du fil pris en concidération. Pour du kevlar de grosse épaisseur, il peut résister jusqu'à des températures atteignant les 200\degree. Le transfert de chaleur au sein du kevlar se déroule plus lentement que losrqu'on parle de coton ou encore du cuir. Cela peut s'avérer interressant dans des utilisations qui nécessite par exemple de transporter des objets ou autre à haute température.

Cependant, le coton est toujours utilisé pour une gamme moyenne de température car il dissipe mieux la chaleur et est également une fibre naturelle qui "respire" mieux que le kevlar. En matière de résistance à la chaleur, on peut dire que le kevlar et le coton se valent, cependant le kevlar offre en plus une meilleure résistance à l'enflammement.

\section{Sites}
-http://www.ansellpro.com/auto/faq2.asp
-http://www.dupont.com/products-and-services/fabrics-fibers-nonwovens/fibers/articles/kevlar-properties.html
-http://www.explainthatstuff.com/kevlar.html

%Geo's part
\section{Production}


La production du Kevlar comprend deux grandes étapes :

\begin{itemize}
	\item la polymérisation, à savoir la production du polymère à partir duquel le Kevlar est fait. Le polymère est synthétisé à partir de deux monomères : la \textit{p-phénylènediamine} ainsi que le \textit{chlorure de téréphtaloyle}. Cette étape est chimique.
	\item le filage du polymère, le transformer en une fibre solide, résistante. Cette phase est mécanique, il s'agit de manipuler le polymère afin de changer ses propriétés mécaniques.
\end{itemize}
	
	
\subsection{Polymérisation}

La réaction de la première étape est une polycondensation\footnote{Durant une réaction de condensation, deux molécules, ou groupements, se combinent pour former une molécule plus grosse tout en éliminant une plus petite molécule (le sous-produit, le plus souvent de l'eau). Chaque étape d'une polycondensation est une réaction de condensation.} réalisée à basse température en milieu solvant, qui aboutit à une longue chaîne de molécules, un polymère. La température de réaction se situe entre -15 et 30\celsius ~et la solution est continuellement remuée durant la polymérisation, pouvant aller de 2 à 24 heures. Le polyamide résultant est composé de noyaux aromatiques (benzène), alternés de groupements aromatiques. Initialement, le solvant était du hexamethylphosphoramide (HMPA) mais pour des raisons de toxicité (mineures), il a été remplacé par une solution de N-méthyl-2-pyrrolidone (NMP) et de chlorure de calcium.
\begin{figure}
\centering
\includegraphics[scale=0.215]{Schema/Kevlar_chemical_synthesis.png}
\caption{Synthétisation du kevlar}
\label{kevlarsynthesis}
\end{figure}
Le sous-produit de cette réaction est de l'acide chlorhydrique, \ce{HCl}. La présence de cet acide en solution pourrait provoquer des réactions de corrosion et endommager du matériel. Il est donc nécessaire d'introduire des bases (généralement des sels de lithium et carbone) pour neutraliser l'acidité.

Après cette étape, la solution à pris la forme d'une masse visqueuse semblable à du gel. Dans cette masse, il pourrait y avoir des sels insolubles en plus de la fibre voulue. Il faut donc les éliminer avant de continuer le processus.
		
\subsection{Filage}
	
A la sortie de la première étape, les fibres sont alignées aléatoirement. Cependant, la Kevlar à un comportement nématique, comme un cristal liquide. C'est-à-dire que les fibres ont tendance à s'aligner dans la même direction, par petits paquets, il y a un certain degré d'ordre naturel. En augmentant la concentration, le degré d'alignement augmente mais n'est pas encore parfait.

\begin{wrapfigure}{R}{0.18\textwidth}
 \vspace{-20pt}
\centering
\includegraphics[scale=0.6]{Schema/kevlar_1.jpg} 
\vspace{-10pt}
\caption{Filage}
\vspace{-20pt}
\label{filage} 
\end{wrapfigure} 


Pour remédier à celà vient la deuxième étape. Les fibres sont filées, lors d'un processus de filage humide. Les fibres sont dissoutes dans de l'acide sulfurique, \ce{H_2SO_4}, et la solution passe sous pression (pour augmenter la concentration) par de très petits orifices afin d'aligner le plus possible les fibres. Le dispositif est une filière. Lors du passage de la solution à travers les minuscules trous de la filière, les fibres s'alignent dans une même direction et le résultat est ensuite refroidi dans de l'eau froide. 

A la fin de cette étape, le résultat est un long fil de Kevlar. Les fils peuvent être utilisés comme tels, comme fil de pêche ou fil à coudre. Ils peuvent aussi être tressé pour former des câbles, des cordes. Une autre manière de les utiliser est de les tisser afin de créer des bandes pouvant entrer dans la composition de gilets pare-balles, gants résistants au feu,...


Sur la figure~\ref{Flowsheet}, on peut voir la flowsheet du procédé.
%	%Définir les formes
%	\tikzstyle{block} = [rectangle, draw, fill=blue!20,text width=14em, text badly centered, rounded corners, minimum height=4em, minimum width=23em, node distance=3cm] %Etape
%	\tikzstyle{line} = [draw, ->,>=latex] %Flèche
%	\tikzstyle{inn} = [ draw, ellipse,fill=green!60, minimum height=2em, align = center] %Entrée
%	\tikzstyle{outt} = [midway, draw, ellipse,fill=red!60, minimum height=2em, align = center] %Sortie
%	\tikzstyle{inout} = [midway, draw, ellipse,shade, top color = red!70, bottom color = green!70, minimum height=2em, align = center] %Entrée/Sortie
%	\tikzstyle{values} = [rectangle, draw, fill = gray!20] %Valeur
%	\tikzstyle{final} = [midway, diamond, draw, fill=yellow!60, text width=4.5em, text badly centered, node distance=3cm, inner sep=0pt] %Produit final


\begin{center}
\begin{wrapfigure}{L}{0.25\textwidth}
 \vspace{-20pt}
\centering

	\scalebox{0.6}{
	\begin{tikzpicture}
	    %Place nodes
	    
	    %Première étape : polymérisation 
	    \node[block] (Pol)at(0,13) {\textbf{Polymérisation} \\
	    \textit{Processus chimique}
	    $$\ce{[C_6H_4(NH_2)_2 + C_8H_4Cl_2O_2]_n} $$ 
	    $$\ce{->[\ce{-2 n HCl}]}$$
	    $$\ce{[-CO-C_6H_4-CO-NH-C_6H_4-NH-]_n} $$
	    };
	    
	    %Deuxième étape : filage
	    \node[block] (Fil)at(0,6) {\textbf{Filage} \\
	    \textit{Processus mécanique}
	    \ce{[-CO-C_6H_4-CO-NH-C_6H_4-NH-]_n} \\
	    $\downarrow$ \\
	    Kevlar
	    };
	
	    \node (in1)at(-5,19) {}; %Premier réactif -4
	    \node (in2)at(5,19) {};  %Deuxième réactif 4
	    \node (out1)at(10.5,13) {}; %Rejet de HCl 9
	    	\node (pfinal)at(0,0) {};  %Produit final
	    	\node[values] (outtt)at(7.2,14) {$306,723\kilogram = 8,404\kilo\mole$}; %Valeurs sur HCl 6.46 13.75
	    		    	
	    % Draw edges
	    
	    %Premier réactif
	    \path [line] (in1) -- node[inn] {p-phénylènediamine \\ \ce{C_6H_4(NH_2)_2}} node[values, near start] {$453,782\kilogram = 4,202\kilo\mole$} (Pol); 
	    %Deuxième réactif
	    \path [line] (in2) -- node[inn] {chlorure de téréphtaloyle \\ \ce{C_8H_4Cl_2O_2}} node[values, near start] {$852,941\kilogram = 4,202\kilo\mole$}(Pol); 
	    %Entre les deux étapes
	    \path [line] (Pol) -- node[inout] {PPD-T \\ \ce{[C_6H_4(NH_2)_2 + C_8H_4Cl_2O_2]_n}} node[values, near start] {$1000\kilogram = 4,202\kilo\mole$} (Fil); 
	    %Rejet de HCl
	    \path [line] (Pol) -- node[outt] {acide chlorhydrique \\ \ce{HCl}} (out1); 
	    %Produit final
	    \path [line] (Fil) -- node[final] {\textbf{Kevlar}} node[values, pos = 0.8] {$1000 \kilogram = 4,202\kilo\mole$} (pfinal);  

	\end{tikzpicture} }
	%\vspace{-5pt}
\caption{Flowsheet}
\vspace{-30pt}
\label{Flowsheet} 
\end{wrapfigure} 
\end{center}

\subsection{Coûts}

Le Kevlar est relativement cher à produire, bien qu'il soit difficile de trouver une estimation précise des coûts de production. Ce coût élevé est dû à différents paramètres. Tout d'abord, il y a derrière le Kevlar des années de recherches et d'innovation. De plus, la réaction nécessite des équipements bien spécifiques, et, par conséquent, peu répandus et chers. Enfin, une troisième cause du haut prix du Kevlar provient de l'étape de filage. En effet, le processus requiert l'utilisation d'acide sulfurique, qui, en plus d'être très coûteux, est extrêmement dangereux. Cela hausse le coût en ajoutant encore une contrainte au niveau des équipements nécessaires, qui doivent résister à cette substance et en exigeant des mesures de sécurité supplémentaires.

Le prix de vente du Kevlar est approximativement compris entre 20 et 40\euro ~du \unit{\cubic\meter}, voire plus, tout dépend de la qualité et du type de Kevlar.
%Léa's part
\chapter{Structure chimique}

Le Kevlar est formé de longues chaines de molécules alignées parallèlement les unes aux par rapport autres procurant ainsi une forte résistance à la traction. 
Cette rigidité est due à la structure de base du kevlar, en effet elle vient de la position en para du noyau benzénique qui est le squelette de la molécule.
Une autre propriété du Kevlar qui contribue à sa résistance en traction sont les liaisions hydrogènes\footnote{La liaison hydrogène est une attraction dipôle-dipôle forte. Elle se présente lorsqu'on a un (ou plusieurs) atome d'hydrogène lié à un atome fortement électromagnétique. C'est une liaison forte.} qui se passent entre l'hydrogène chargé positivement et le groupe carbonyle.
\section{Liaison hydrogène}
Le groupe carbonyle ainsi que l'hydrogène sont présent dans chaque monomère de ce fait, les liaisons hydrogènes se font de mainière récurente et multiple à travers les longues chaines de molécules.

Cependant ces liaisions hydrogènes sont aussi une faiblesse du Kevlar, en effet en conctact de l'eau ces liasions ont tendance à se casser pour priviligier des lisasion hydrogènes entre l'hydrogène du Kelvlar et l'oxygène de l'eau.



\end{document}